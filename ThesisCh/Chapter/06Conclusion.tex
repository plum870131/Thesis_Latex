\chapter{結論}
\label{chp:6}


\section{研究總結}

本研究為了能夠達到一單點對單點的三維定位系統,且需具有靈活安裝於不同環境與位置的特性,從分析不同室內定位技術與方法的特色開始,將目光著重在具有靈活度潛力的LED與PD系統上。其中,為了解決此領域中的應用情境過多的限制,本研究於第三章建立一廣用於三維空間中的定位演算法,其可以不限制量測者與目標物平面平行,又能達到三維定位,且完整考量朗博次方,提供LED與PD系統更具靈活度的應用可能。

除了建立一廣用於三維空間中的定位演算法,本研究於第四章建立一多LED對多PD的定位系統的模擬系統,可透過簡單的調整參數,對不同次系統規格、不同使用情境與ROI、以及不同誤差模擬方式,進行系統成效的評估。由此系統成效評估方式,於\ref{chp:design_result}章中,簡單對特定使用情境分析各項次系統規格設計變數的影響,包含朗博次方、硬體指向中得天頂角與硬體數量。對各項變數與系統成效的關係有基本了解之後,本研究於第五章中建立一最佳化流程,針對不同情境進行硬體總數、朗博次方與各硬體指向的最佳化,提供系統設計者最合適的次系統規格,也讓使用者在實際進行硬體系統建制之前,得以對系統表現有基本的理解。

% 本研究建立一分析流程,將資料數量龐大的模擬模型軌跡與有限資料點的真實系統軌跡,經前處理後獲得兩等量等距結果,並進一步透過位置資訊序列與角度資訊序列,建立仿射轉換與輪廓差異指標後,得到單一複合指標,量化模擬模型與真實系統軌跡,提供模型驗證更多有效資訊進行調整。並透過一三輪車模型案例,利用閉迴路系統產生模擬模型軌跡,以及開迴路系統加入偏差參數獲得真實系統軌跡,計算差異量化指標後,以有效替代模型進行參數估測,驗證該複合指標的有效性,同時,對於有雜訊軌跡提供調整建議,提升方法可行性。

% 越接近真實系統的電腦模型,能使工程師更加掌握真實系統樣貌。而在模型驗證領域,對於軌跡動態輸出量化有其必要性,透過量化指標可以定量地衡量系統與參數間的關係,並得到量化差異結果進一步修正系統或調整參數。本研究提出的複合量化指標,可以針對不同操作軌跡模型進行差異量比較,也在實驗案例中,提出替代模型的概念,進一步使用量化指標達到參數估測的目的。

\section{未來目標}


\begin{description}

    \item[- 提升演算法靈活度:] 第\ref{chp:3}章提出的演算法中,仍然有限制硬體種類需相同,侷限了硬體挑選以及系統設計上的自由度。透過調整演算法可解除此限制,繼續提升演算法靈活度,然而演算法複雜度也隨之提升,需特別注意。
    
    \item[- 提升模擬真實度:] 本研究與大多LED對PD的定位系統相同,在模擬中並沒有考量多重路徑誤差、干涉等效果的誤差,因此不夠貼近實際系統的表現。無論是以實驗評估誤差,或是將模擬系統加入更多誤差考量,便可檢考模擬與實際系統的差異。
    
    \item[- 硬體驗證:]本研究與大多LED對PD的定位系統相同,缺乏硬體驗證,僅利用模擬評估演算法與誤差;然而模擬與實際情況必有出入,最正確評估系統的方式便需架設完整的光通訊系統,並利用該系統進行定位,針對實驗結果再進行分析與研究以調整演算法或是系統設計。
    
    

\end{description}
