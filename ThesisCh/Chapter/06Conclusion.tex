\chapter{結論}
\label{chp:6}


\section{研究總結}


為了用可攜式單位達到一單點對單點的三維定位系統,以達到靈活安裝於不同環境與位置的需求,本研究由探討不同室內定位技術與方法的特性開始,根據需求將目標聚焦在具有以可攜式單位進行三維定位潛力的LED與PD定位系統上。然而,此領域中現有文獻大多利用情境限制與硬體組態限制來簡化系統複雜度,例如限制量測者與目標物平面平行、忽略朗博次方、限制硬體擺設方法等。這些限制除了降低了次系統設計的自由度外,也使應用情境大大的受限,與我們的目標不符。因此,為了解決此領域中應用情境與次系統設計過多的限制,本研究於第三章建立一廣用於三維空間中的三維定位演算法,在不限制量測者與目標物平面平行的情況下,達到三維定位並可得到目標物姿態,且該演算法完整考量朗博次方,並將LED與PD的硬體數量與各硬體的擺設指向視為變數,提供LED與PD系統更具靈活度的應用可能。

除了建立一廣用於三維空間中的定位演算法,本研究於第四章建立多LED對多PD的定位模擬系統,可透過簡單的調整參數,以對不同次系統規格、不同使用情境與ROI、以及不同誤差模擬方式,進行系統成效的量化評估。為了凸顯本研究演算法與其他文獻之差異,於\ref{chp:design_result}章中提出一使用情境,其擁有$3\times 3\times 3m$的平移樣本空間與61種旋轉姿態作為樣本點,以此模擬目標平面與量測平面不需平行之移動行為。透過分別調整不同次系統設計參數、誤差模擬等參數,我們可以分析各參數對此使用情境系統效能的影響;除了該情境外,也於\ref{chp:scene_effect}中提出另外三個使用情境,得到不同使用情境會有不同系統成效的結論且各參數的影響也不盡相同,系統的複雜度很高,也凸顯了針對不同情境進行最佳化有其必要性。

因此,對各項變數與系統成效的關係有基本了解之後,本研究於第五章中建立一最佳化流程,針對不同情境進行次系統設計的最佳化,透過指定硬體數量的設計空間,將設計空間中的朗博次方與各硬體的指向進行最佳化,提供針對該情境最合適的次系統規格,讓使用者在實際進行硬體系統建立之前,得以對系統表現有基本的理解,並以此最佳化結果作為硬體系統搭建的參考依據。本章節使用三個使用情境作為最佳化案例,以最佳化結果來看,各情境中得出的最佳次系統設計十分不同,也顯示出次系統設計的重要性。

總而言之,LED與PD的定位方法具有以可攜式單位進行三維定位的潛力,而系統具有靈活度的同時複雜度也高,透過改變次系統設計、改變使用情境、改變模擬參數,都會使定位成效有所不同。因此,提出廣用於三維空間的定位演算法是為提供使用者一單點對單點進行三維定位的可能性;而模擬與評估方法提供了一有效率的方法對此複雜的系統進行量化評估,讓使用者得以用方便的模擬軟體對複雜的系統有所了解;至於最佳化方法則是針對特定使用情境提供最合適的次系統設計,除了可以了解此使用情境的極限外,也能作為硬體搭建的設計參考。


% 本研究建立一分析流程,將資料數量龐大的模擬模型軌跡與有限資料點的真實系統軌跡,經前處理後獲得兩等量等距結果,並進一步透過位置資訊序列與角度資訊序列,建立仿射轉換與輪廓差異指標後,得到單一複合指標,量化模擬模型與真實系統軌跡,提供模型驗證更多有效資訊進行調整。並透過一三輪車模型案例,利用閉迴路系統產生模擬模型軌跡,以及開迴路系統加入偏差參數獲得真實系統軌跡,計算差異量化指標後,以有效替代模型進行參數估測,驗證該複合指標的有效性,同時,對於有雜訊軌跡提供調整建議,提升方法可行性。

% 越接近真實系統的電腦模型,能使工程師更加掌握真實系統樣貌。而在模型驗證領域,對於軌跡動態輸出量化有其必要性,透過量化指標可以定量地衡量系統與參數間的關係,並得到量化差異結果進一步修正系統或調整參數。本研究提出的複合量化指標,可以針對不同操作軌跡模型進行差異量比較,也在實驗案例中,提出替代模型的概念,進一步使用量化指標達到參數估測的目的。

\section{未來目標}


\begin{description}

    \item[- 提升演算法靈活度:] 第\ref{chp:3}章提出的演算法中,仍然有限制硬體種類需相同,侷限了硬體挑選以及系統設計上的自由度。透過調整演算法可解除此限制,繼續提升演算法靈活度,然而演算法複雜度也隨之提升,需特別注意。
    
    \item[- 提升模擬真實度:] 本研究與大多LED對PD的定位系統相同,在模擬中並沒有考量多重路徑誤差、干涉等效果的誤差,因此不夠貼近實際系統的表現。無論是以實驗評估誤差,或是將模擬系統加入更多誤差考量,便可檢考模擬與實際系統的差異。
    
    \item[- 硬體驗證:]本研究與大多LED對PD的定位系統相同,缺乏硬體驗證,僅利用模擬評估演算法與誤差;然而模擬與實際情況必有出入,最正確評估系統的方式便需架設完整的光通訊系統,並利用該系統進行定位,針對實驗結果再進行分析與研究以調整演算法或是系統設計。
    
    

\end{description}
