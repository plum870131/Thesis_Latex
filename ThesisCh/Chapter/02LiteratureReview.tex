\chapter{文獻回顧}




\cite{radiometry_and_photometry}


本研究所探討情境下的量測方法有以下需求:有易於拆裝的量測單位與被量測單位,能夠快速進行拆裝、靈活應用於不同場域,而安裝完成後即可進行三維相對位置量測。為滿足靈活且易於安裝且能夠應用在不同場域的特性,需要有體積小、能耗低、所需校正步驟少、能夠使用於不同場域等特色。

因此,本章節先定義所謂相對定位,再介紹現有文獻的定位技術與方法,比較優缺點並凸顯「近紅外光定位」的優勢,近而針對本論文著重的「LED與PD以近紅外光波段進行定位」的文獻進行探討,敘述此領域研究現況與困難。







% 重申自己主要的目標:(只有LED-PD近紅外光波段符合的原因)
% 低成本、不需預先了解環境資訊、裝設範圍小、方便架設、速度

\section{相對定位定義}

% - 問題定義:(利用數學符號描述相對定位問題的定義(軌跡、時間等))
    
    在開始進入文獻探討之前,需先以數學定義何謂本論文所欲量測之「相對定位」。首先,本論文所討論的情境為一量測物針對另一特定目標物進行相對位置的量測,如智慧工廠內的機械手臂欲取得與移動載具之間的相對關係,以利夾取搬運物品至載具上進行運送。
    
    我們將取得相對位置的一方稱為量測者,如案例中的機械手臂;而量測者所欲取得相對位置的特定物體稱為目標物,如案例中的移動載具;兩者皆為剛體。因此,可以將量測者與目標物各自視為兩移動座標系,兩者在空間中各自有位置、旋轉的六個自由度,可以利用齊次座標轉換(Homogeneous Transformation)表示座標系之間的平移與旋轉
  
    \begin{equation}
        \label{eqn:homogeneous}
        c=c
    \end{equation}
    
    在空間中,任意座標系有六個自由度,包含位置的三個自由度以及旋轉的三個自由度,而任兩座標系之間的平移關係即為相對位置資訊。本論文研究目標為量測一觀測者對一被觀測者的相對位置,在此定義兩座標系分別為觀察者座標系與被觀察者座標系,兩座標系上各自裝載者傳感器與訊號發送器,觀察者座標系上藉由傳感器量測被觀察者座標系上訊號發送器所傳送之訊號,計算出目標物座標系相對觀察者座標系的平移位置。
    
    而無論是傳感器或是訊號發送器皆為固定。
    
    其中無論觀測者或被觀測者皆為剛體,而傳感器與訊號發送器裝載於剛體座標系上。

首先從量測波段切入,縮小至光量測

比較光量測的不同方法,縮小至我在做的部分

進行這個領域的比較

