\chapter{室內定位相關文獻探討}







統整本研究探討的情境(參考\ref{chp:motivate}章)有以下需求:硬體系統需為低能耗,可封裝成量測單位與目標物單位,進行單點對單點的定位。本章節將介紹定位領域中的不同技術與方法,於\ref{chp:relative}章先定義相對定位,依照圖\ref{pic:intro_pos}中的分類,依序介紹定位方法\ref{chp:method}章與定位技術\ref{chp:technique}章,比較各自的優缺點,根據上述需求聚焦在「LED與PD定位技術」,\ref{chp:LEDandPD}章便針對「LED與PD定位技術」進行深入探討,敘述此領域研究現況與困難。








% % 重申自己主要的目標:(只有LED-PD近紅外光波段符合的原因)
% % 低成本、不需預先了解環境資訊、裝設範圍小、方便架設、速度

\section{相對定位定義}
\label{chp:relative}
% -(利用數學符號描述相對定位問題的定義(軌跡、時間等))
    
    在開始進入文獻探討之前,需先以數學定義何謂本論文所欲量測之「相對定位」。我們將取得相對位置的一方稱為量測者,如案例中的機械手臂;而量測者所欲取得相對位置的特定物體稱為目標物,如案例中的移動載具;兩者皆為剛體。

    \begin{figure}[h]
        \centering
        \includegraphics[width=9cm]{ch2pic/homo_trans.png}
        \caption{LED座標系與PD座標系及相對關係}
        \label{pic:homo_trans}
    \end{figure}
    
    因此,可以將量測者與目標物各自視為兩移動座標系如圖\ref{pic:homo_trans},兩者在空間中各自有位置、旋轉的六個自由度,可以利用齊次座標轉換(Homogeneous Transformation)表示座標系之間的平移與旋轉(式\ref{eqn:homogeneous}),其中$^{L}\boldsymbol{P}$為LED的位置,而$^{PL}\boldsymbol{P}$為投影至PD座標系上的LED位置,描述投影的矩陣為齊次轉換矩陣$^{PL}\boldsymbol{H}$,左上標$PL$表示將LED座標系轉換至PD座標系,而$^{PL}\boldsymbol{T}$與$^{PL}\boldsymbol{Ro}$各自代表平移與旋轉的轉換矩陣,符號可參考符號列表(第\pageref{chp:symbol}頁)。
    
  

    \begin{equation}
        \label{eqn:homogeneous}
        \begin{aligned}
        {\left[\begin{array}{c}
        { }^{P L} \boldsymbol{P} \\
        1
        \end{array}\right]={ }^{P L} \boldsymbol{H}\left[\begin{array}{c}
        { }^{L} \boldsymbol{P} \\
        1
        \end{array}\right] } &=\left[\begin{array}{cc}
        { }^{P L} \boldsymbol{R} \boldsymbol{o} & { }^{P L} \boldsymbol{T} \\
        0 & 1
        \end{array}\right]\left[\begin{array}{c}
        { }^{L} \boldsymbol{P} \\
        1
        \end{array}\right] \\
        &=\left[\begin{array}{cccc}
        { }^{P L} \gamma_{11} & { }^{P L} \gamma_{12} & { }^{P L} \gamma_{13} & { }^{P L} {t}_{x} \\
        { }^{P L} \gamma_{21} & ^{P L } \gamma_{22} & { }^{P L} \gamma_{23} & { }^{P L} {t}_{y} \\
        { }^{P L} \gamma_{31} & ^{P L} \gamma_{32} & { }^{P L} \gamma_{33} & { }^{P L} {t}_{z} \\
        0 & 0 & 0 & 1
        \end{array}\right]\left[\begin{array}{c}
        { }^{L} {x} \\
        { }^{L} {y} \\
        { }^{L} z \\
        1
        \end{array}\right]
        \end{aligned}
    \end{equation}


    
    
   其中,兩座標系之間的平移關係$^{PL}\boldsymbol{T}$即為欲得到的相對位置資訊,共有三個自由度。以上皆以卡氏座標系(Cartesian Coordinate)表示,然而在定位中也常使用球座標系表示,讓定位以「距離$d$」與「方位($\alpha,\beta$)」呈現,較為直觀,可將方位想像成「看哪裡」,而距離則為「看多遠」。兩座標系之間的換算如式\ref{eqn:car2sph},球座標系中的位置$\boldsymbol{P}^{sph}$透過公式轉換為卡式座標系表示的$\boldsymbol{P}$:

   \begin{equation}
    \label{eqn:car2sph}
        \begin{aligned}
        \boldsymbol{P}^{sph}&=\left[\begin{array}{c}
            d\\\alpha\\\beta
            \end{array}\right] \\  
        \boldsymbol{P} &= \left[\begin{array}{c}
            x\\y\\z
            \end{array}\right]=
            \left[\begin{array}{c}
            d\sin\alpha\cos\beta \\
            d\sin\alpha\sin\beta\\
            d\cos\alpha
            \end{array}\right] 
    \end{aligned}
   \end{equation}

   在此需特別注意的是,LED座標系與PD座標系指的是「兩空間中移動的座標系」,而卡氏座標系與球座標系則是座標系的呈現方式;LED座標系可以用卡式座標系的$[x,y,z]^T$表達,也可以用球座標系的$[d,\alpha,\beta]^T$表示,PD座標系亦然。



\section{後處理方法分類}
\label{chp:method}

    從定位流程圖\ref{pic:intro_pos}來看,目標物會主動或是被動的將訊號傳送至量測者硬體,量測者接收到訊號之後再以不同的方法進行後處理,計算出目標物與觀察者之間的相對關係。

    \begin{figure}[h]
        \centering
        \includegraphics[width=14cm]{ch1pic/intro_pos.png}
        \caption{常見的室內定位流程與分類}
        \label{pic:intro_pos}
    \end{figure}
    
    後處理方法可以分成兩部分:室內定位所使用的硬體在接收訊號時,首先會根據量測者所接收的資訊種類有所不同;接收資訊後,如何利用資訊計算出相對位置,則是屬於定位演算法的部分。
    
    

    \subsection{資訊種類}
    
    \begin{figure}[h]
        \centering
        \includegraphics[width=12cm]{ch2pic/method_sort.png}
        \caption{資訊種類與演算法之間的關聯(參考\cite{survey_light2018})}
        \label{pic:method_sort}
    \end{figure}

    訊號強度(Received Signal Strength,簡寫RSS)為最常見的資訊種類\cite{survey_indoor2018},使用的是該硬體所接收到的訊號強度,而利用訊號強度的演算法也十分多樣,包含多點定位、三角法、指紋法等,可參考\ref{chp:method-algorithm}章。

    時間資訊種類則是硬體利用訊號之間的時間差,量測距離的方法,常見使用的技術為紅外線,紅外線可以利用訊號傳送到不同接收子的時間差,判斷出之間的距離。若要利用時間資訊,在各硬體之間則需要將時間同步,在實作上為一大挑戰\cite{survey_indoor2014}。

    若獲取的是角度資訊,基本上演算法便會搭配三角法,綜合多個接收子的角度獲得相對位置。影像則屬於相機獨有,且搭配的定位演算法為影像辨識。


    \subsection{定位演算法}
    \label{chp:method-algorithm}
    常見的演算法種類如下:
    \begin{description}
        \item[- 多點定位(Multilateration)]\hfill 
        
        \qquad
        多點定位如圖\ref{pic:multilateration}所示,需在環境中建立多個參考點並固定位置,量測目標物與多個參考點之間的距離,進而以各參考點為中心、距離為半徑畫圓求交點。

        \begin{figure}[h]
            \centering
            \includegraphics[width=6cm]{ch2pic/multilateration.png}
            \caption{多點定位法\cite{survey_light2020}}
            \label{pic:multilateration}
        \end{figure}
    
        \item[- 三角測量(Triangulation)] \hfill 
        
        \qquad
        三角測量法如圖\ref{pic:triangulation}所示,需要在環境中建立多個參考點,藉由量測目標物與各參考點之間的角度關係,推算目標物位置。
        \begin{figure}[h]
            \centering
            \includegraphics[width=9cm]{ch2pic/triangulation.png}
            \caption{三角測量法\cite{survey_light2018}}
            \label{pic:triangulation}
        \end{figure}
        
        \item[- 指紋法(Fingerprinting)] \hfill 
        
        \qquad
        指紋法在環境中建立多個參考點,在實際進行測量前,需進行事先測量訓練的階段。在訓練階段時需將目標物在環境中移動,蒐集大量訊號數據庫,量測階段則藉由接收訊號與訓練階段所建立的數據庫參照,尋找最有可能存在的位置\cite{survey_light2020},近年則是有許多以機器學習方法增加此演算法的精確度。
        \begin{figure}[h]
            \centering
            \includegraphics[width=13cm]{ch2pic/fingerprinting.png}
            \caption{指紋法流程圖\cite{pic:fingerprinting}}
            \label{pic:fingerprinting}
        \end{figure}
        
        \item[- 影像辨識]\hfill 
        
        \qquad
        三維空間中的物體,透過相機拍攝成為二維的影像時,為三維空間投影至二維影像平面上的幾何轉換。利用影像辨識求解相對位置時,則是試圖將二維影像中的特徵點比對、轉換回三維空間中,利用PnP演算法(Perspective-n-Point)解出目標物相對相機座標系的位置與方位\cite{pic:image_processing}。
        
        \begin{figure}[h]
            \centering
            \includegraphics[width=15cm]{ch2pic/image_processing.png}
            \caption{影像辨識示意圖\cite{pic:image_processing}}
            \label{pic:image_processing}
        \end{figure}
        
        \qquad
        為了使辨識過程中,辨識特徵點的難度降低,通常會設計特徵明顯的標記,例如常見OpenCV中的ArUco標記,利用其快速的辨識ID以及距離、方位等資訊。
        
        \begin{figure}[h]
            \centering
            \includegraphics[width=5cm]{ch2pic/aruco.png}
            \caption{ArUco標記例子\cite{pic:aruco}}
            \label{pic:aruco}
        \end{figure}
        
        \item[- 幾何方法] \hfill 
        
        \qquad
        此類方法常見於LED與PD的定位方法,並沒有一通用的分類名稱,其接收的訊號強度與距離與角度皆有關,因此需綜合多點定位與三角定位,透過訊號強度與相對位置之間的幾何關係推算相對位置\cite{survey_light2020}。
        
        \item[- 其他]\hfill 
        
        \qquad
        常見的其他種類演算法多會使用混合系統(Hybrid System)\cite{survey_indoor2018},除了使用量測距離、角度等位置資訊的感測器,多與其他量測姿態、速度的感測器結合使用,如動作捕捉(Motion Capture)領域中使用慣性量測單元(Inertial Measurement Unit,簡稱IMU)量測點的移動軌跡。

    \end{description}

    \subsection{小結}

    根據\ref{chp:motivate}章所總結對靈活度等的需求,演算法的影響較為顯著,而其中幾何方法是最合適的,因為此方法主要利用的為「兩者之間的幾何關係」,較能夠適應不同環境、不同擺放位置的改變,做到「一對一的定位」。相較起來,多點定位、三角法、指紋法都需要多個固定點的參考點的量測環境,無法符合易於安裝單元的需求,可以想像成「環境對一點的定位」。



\section{定位技術分類}
\label{chp:technique}

    室內定位所使用的技術(Technique)一樣十分多樣,非電磁波段的定位主要為超聲波,應用訊號發射與接收之間的時間差,推算與目標物之間的距離。然而該技術受溫度影響,且對目標物的辨識能力不佳,因此目前著重在自駕車與載具中障礙物的有無偵測上\cite{survey_ultrasonic},並不符合研究目標,所以以下章節聚焦在電磁波段內的定位進行分析。

    電磁波段內有許多不同現今受到關注的量測技術,電磁波段內常見的技術於\ref{chp:intro}章內有簡單介紹,而電磁波以光速傳播,擁有高傳播速度與不受介質溫度與濕度影響的特性,減少可能對量測訊號造成影響的因素。

    \begin{figure}[h]
        \centering
        \includegraphics[width=12cm]{ch2pic/electro_spectrum.png}
        \caption{電磁波頻譜度\cite{Spectrum}}
        \label{pic:spectrum}
    \end{figure}

    我們由電磁波的頻率切入進行分類,頻譜圖如圖\ref{pic:spectrum},能夠用於定位的波段為頻率低於約790THz的波段,因為高於此波段的電磁波具有過高能量,電磁波頻率越高所含能量越高使高頻波段對人體有害,因此此波段中的X光、紫外線、伽瑪射線無法使用於日常應用,不予考慮。

    對人體無害的波段則分為兩部分探討:\ref{chp:radio}章探討頻率低於300GHz的低頻波段\cite{book_electromagnetic},使用此波段的技術包含Wifi、藍芽、RFID;而在\ref{chp:light}章介紹頻率介於790THz與300GHz之間的光波段,此波段的技術包含紅外線、可見光定位以及相機定位。以下章節便以圖\ref{pic:electro_sort}為架構依序介紹。


    \begin{figure}[h]
        \centering
        \includegraphics[width=13cm]{ch2pic/electro_sort.png}
        \caption{以電磁波頻率分類定位技術}
        \label{pic:electro_sort}
    \end{figure}

    











    \subsection{無線電波段定位}
    \label{chp:radio}
    
        低頻波段也就是俗稱的無線電波(Radio Waves),應用發展歷史長,對於可使用的頻率有嚴格規範,台灣由國家通訊委員會訂定嚴格的可使用頻率波段\cite{rf_law},保證軍警、醫療、廣播等的傳播需求。常見的定位技術包含室內與多數裝置既有的WiFi與藍芽裝置,或是較新的RFID定位技術。
    
        無線電波的主要特色包含可穿透大多障礙物,因此可用於跨房間非可視範圍(NLoS)的定位\cite{survey_indoor2018},增加了應用場域,然而也大大提升了訊號處理的難度,感測器難以辨別訊號衰減原因為距離、角度、亦或是障礙物,因此選擇非可視範圍內定位即捨棄精度。除此之外,無線電波的傳遞距離很長,定位範圍較廣,WiFi甚至可達距離50公尺的定位\cite{survey_indoor2014}。所需面對的誤差處理包含電磁干擾、同頻道干擾(Co-channel Interference)、受障礙物與金屬影響的穿透效果等。

        至於無線電波段所使用的定位演算法,大多系統皆是利用指紋法、多點定位、三角法的方式,共通點是都需要多個參考點,事先的安裝複雜,應用場域著重在固定場域,而精度多為公尺量級。
    
        舉例來說,此波段大多應用在固定的場域中,方法包含指紋法的建立數據庫,或是利用大量的感測器與訊號發射器來判斷定位\cite{survey_rfid},系統設置皆與圖\ref{pic:rfid_system}相似,差異於感測器擺放的方法以及演算法。其中,最經典的LANDMARC例子\cite{landmarc}的環境如圖\ref{pic:landmarc}所示,也是將感測器散布於環境中。

        

        \begin{figure}[h]
            \centering
            \includegraphics[width=6cm]{ch2pic/rfid_system.png}
            \caption{RFID定位系統示意\cite{survey_rfid}}
            \label{pic:rfid_system}
        \end{figure}

        \begin{figure}[h]
            \centering
            \includegraphics[width=8cm]{ch2pic/landmarc.png}
            \caption{LANDMARC系統\cite{landmarc}}
            \label{pic:landmarc}
        \end{figure}

        無線電波段也有少數例用\ref{chp:method-algorithm}章中所提到的三角法,如\cite{case:rfid_1to1}研究(圖\ref{pic:rfid_1to1}),例用RFID天線所接收到的訊號強度與入射角度的關係,在載具上放置兩天線,而擺放角度有90度的差異,藉由RFID發射器傳送訊號至天線的入射角度不同,計算出目標物的方位。此文獻提出了一種一對一定位的方法,然而其只能解出目標物的方位,也就是二維定位中的角度,並不包含距離。除此之外,由於可穿透障礙物的特色,隨著測試環境中添加障礙物、人、移動物品,皆會影響到訊號,造成誤差。

        \begin{figure}[h]
            \centering
            \includegraphics[width=8cm]{ch2pic/rfid_1to1.png}
            \caption{例用三角法定位的RFID系統\cite{case:rfid_1to1}(a)系統架構 (b)載具實驗}
            \label{pic:rfid_1to1}
        \end{figure}
        
        最後總結無線電波段的特色:此波段常使用於傳輸訊號的功能上,因此有成熟的編碼與解碼技術,能夠輕易達到辨識目標物的功能;而能夠穿透障礙物的特性,以及傳輸距離廣,使得應用範圍大又不受視線範圍拘束;然而精度相較低,因此大多仰賴多個參考點或是事前建立數據庫的指紋法,應用的靈活度降低,較不符合本研究目標,因此著重在\ref{chp:light}章中的光波段定位。

       



       


        \subsection{光波段定位}
        \label{chp:light}
            
            光波段的定位發展近幾年來十分顯著,主要原因為光學硬體上的進步,促使光通訊(Light Communication)的發展,使發光與感光元件具有傳播ID、訊息等資訊的能力,因此光通訊與光定位便於近年得到研究關注\cite{survey_light2018}。光通訊流程如圖\ref{pic:vlc_flow}所示,利用驅動器調控光源的閃爍頻率、發光強度,來達到編碼的效果,最常見的編碼方式即為關閉調控(On-Off Keying,以下簡稱OOK),藉由開關光源傳送一段二進制訊號;感測器多以PD接收,接收後透過解碼取得資訊\cite{pic:vlc}。
            \begin{figure}[ht]
                \centering
                \includegraphics[width=15cm]{ch2pic/vlc_flow.png}
                \caption{光通訊流程圖\cite{pic:vlc}}
                \label{pic:vlc_flow}
            \end{figure}

            光波段在應用場域有一優勢,其可應用於於無線電波限制的場域,如機場與醫院等特殊醫療場域\cite{case:vlc_mobile},且在如今無線電波訊號充斥環境的狀況下,即使無線電波擁有很廣頻率選擇,然而隨著現進通訊需求的提升,無線電波段已被證實有塞車的狀況出現,而光波段即提供一替代方案,應對此困境。
            
            
            再來,光波段無法穿透障礙物,僅能進行可視範圍內(Line-of-Sight,以下簡稱LoS)的定位,捨棄廣域的定位,獲得較為單純的量測數據。此特性同時增加光通訊的安全性以及穩定度,著重針對可視範圍進行通訊與定位,不必考量其他訊號,也不受多餘訊號的干擾\cite{vlc_adv}。例如載具運行時,僅需著重分析於身旁的其他物體,位於隔壁房間的物體毫不重要,因此可忽略不重要的資訊,增加處理速度與有用訊號穩定度。即便如此,僅能進行LoS定位也限制了能夠量測的範圍,在環境阻擋物較為複雜的環境會擋住光訊號的傳輸,使其無法進行定位。


            除此之外,電磁波有一特性為:波長愈短可達到的量測精度愈高,且感測器與訊號發射器的硬體大小也愈小,例如市售便宜常見的光波段感測器如PD感光二極體尺寸量級為mm或以下\cite{datasheet:led_sfh4545}且能耗低,而使用低頻波段的RFID感測器常見量級為cm\cite{datasheet:rfid_tag}。因此,選擇波長較短的光波段相較無線電波段具有硬體上的優勢。

            \begin{figure}[h]
                \centering
                \includegraphics[width=9cm]{ch2pic/method_compare.png}
                \caption{無線電波段與光波段的比較}
                \label{pic:method_compare}
            \end{figure}


            光波段定位也分為許多種類(參考圖\ref{pic:electro_sort}),以下分別於\ref{chp:light_electro}章探討使用波段的差異,以及於\ref{chp:light_receiver}章介紹感測器硬體技術的選擇。


            \subsubsection{波段分類}

            \label{chp:light_electro}

            參考圖\ref{pic:electro_sort},光波段依照人眼可視與否分成可見光與紅外光兩類,而其中紅外光內由於硬體特性不同,又分為近紅外光與遠紅外光。以下依照不同面向條列討論:

            \begin{description}

                \item[- 人眼可視與否]\hfill 
                
                \qquad
                依照人眼可視區分為可見光與紅外光波段,而紅外光頻率低、能量較小,且於視網膜成像的難度也大,因此較為安全。而可見光的優勢於其可以照明,因此大多可見光定位會應用於需照明的場域,配合照明用的光線進行定位。但凡使用情境不想影響人們的感知,亦或是光源需移動,有直射眼境造成人眼不適的可能時,便會選擇使用紅外光波段。

                \qquad 
                本研究目標需應用於移動物體上,因此紅外光呈現優勢,紅外光無法在視網膜成像的特性讓使用者並不會感受到不適

                \item[- 成本] \hfill 
                
                \qquad
                光波段的光感測器常見的材料有矽(Si)、鍺(Ge)與III或V族元素,價格最便宜的是矽材料,而矽感測器的感光波段約在400-1000nm,也就是可見光與近紅外光(NIR)波段。但凡波長較長的紅外光在硬體材料便需要使用到昂貴的鍺或其他三五族元素\cite{si_pd},這也是為什麼普遍紅外光波段的硬體的印象都較為昂貴,然而近紅外光波段的感測器與可見光價格接近。

                \qquad
                除了硬體本身的成本以外,系統架設難易度也需納入考量。常見可見光定位的文獻會強調系統架設難易度低,因為可以利用室內常見的照明光源,不需額外添置訊號發射器\cite{vlc_adv}。然而需注意的是,可見光定位需利用光通訊分辨ID時,進行編碼需要在光源硬體加裝驅動器,即代表需要將原有燈具改成能夠進行編碼的硬體、加裝驅動器等,所需付出的成本不得被忽視。因此,在架設成本上,我認為波段的無論影響不大。
                
                \qquad
                因此,僅考量成本時,無論是PD還是影像感測器,挑選可見光與近紅外光波段較為合適。
        
                \item[- 環境光源的影響] \hfill 
                
                \qquad
                光波段的誤差來源主要包含多重路徑傳輸(Multipath effect)與環境光源(Ambious Light Source)\cite{survey_light2020},其中多路徑傳輸無關波段,影響程度皆相同。而環境光源的部分,強度過高可造成訊號偏移,甚至硬體飽和導致訊號失真,因此需有效的降低環境光源的強度以保持定位的準確度。
            
                

                \qquad
                環境光源包含了室內的光源以及太陽光源,室內的光源使用交流電的頻率約在120hz,可以針對頻率進行濾波。而太陽光源的強度與電磁波頻率有關,可以由太陽輻射波譜(Solar Irradiance Spectrum)觀察強度與波長的關係。太陽光照射至地球時,傳送至地表的能量與大氣層的吸收度有關,高峰即為可見光波段,低谷則出現在760nm左右的氧氣吸收帶(Oxygen A-band)、與940nm和1550nm附近被水蒸氣吸收之波段\cite{book:solar_spectrum}。
                
                \begin{figure}[h]
                    \centering
                    \includegraphics[width=10cm]{ch2pic/solar_spectra.png}
                    \caption{太陽輻射波譜\cite{astm}}
                    \label{pic:solar_spectrum}
                \end{figure}

                \qquad
                由於利用光波段定位最大誤差來源之一就是要克服其他光所造成的雜訊,因此在選擇工作波長時,需挑選太陽光能量較低的波段,也就是於近紅外光波段中的760nm與940nm,以及遠紅外光波段內的1550nm。
        
                \end{description}
                



                綜上所述,根據本研究的需求,最適合選擇使用近紅外光波段中的太陽能量低谷波段,簡單整理於圖\ref{pic:light_freq_compare},可以同時節省硬體成本、降低環境光源的影響,也能夠不造成使用者的眼經不適。

                \begin{figure}[h]
                    \centering
                    \includegraphics[width=15cm]{ch2pic/light_freq_compare.png}
                    \caption{光波段中的波段特性比較}
                    \label{pic:light_freq_compare}
                \end{figure}


            \subsubsection{接收子類型:PD與影像感測器}
            \label{chp:light_receiver}

                

                \ref{chp:light_electro}章中討論了光波段中不同波段的優缺點,而本小節中將討論光波段中常使用的兩種硬體差別,其分別為PD與影像感測器。

                \begin{description}

                    \item[- 影像感測器]\hfill 
                    
                    \qquad
                    影像感測器包含常見的相機,運作原理為每次取樣時,獲得多個像素的強度資訊,資訊量極多,是眾多定位技術中最消耗運算資源的。其定位方式是利用辨識已知的特徵圖案,進而將圖片中圖案的大小與變形與特徵進行比對,利用PnP眼算法推算目標物的距離以及姿態(參考\ref{chp:method-algorithm}章)。其中,若量測範圍要廣,由於隨距離增加,投射於影像平面上的圖案大小衰減快速,因此特徵圖案則需要越大越好。然而其取樣頻率大多在百赫茲內,且視覺辨識所需要的運算資源多,因此成本高,且運算速度也較慢。

                    \qquad
                    影像感測器所辨識的特徵點又可分為主動與被動傳送訊號的,被動傳送訊號的包含ArUco標誌(參考圖\ref{pic:aruco}),其需透過環境光源照設可見光波,讓影像感測器得以看到標誌,若室內光源未開,則無法定位。反之,主動傳送訊號的標誌便是光源,大多室內定位會使用半徑15公分以上的崁燈,利用相機的滾動式快門效應(Rolling Shutter Effect),將條紋圖案視為特徵辨識。其中,滾動式快門效應會造成條紋圖樣的原因可參考圖\ref{pic:rolling_shutter},因為CMOS感測器並非同時進行所有像素的成像,而是輪流一列一列的成像,搭配光源的閃爍,各列成像時便會各自呈現亮暗,因此產生條紋樣式。除此之外,透過調整光源的閃爍頻率,可以調整條紋的粗細,進而達到編碼與解碼的效果,如圖\ref{pic:rolling_shutter_case}。

                    \begin{figure}[h]
                        \centering
                        \includegraphics[width=8cm]{ch2pic/rolling_shutter.png}
                        \caption{滾動式快門效應的成因\cite{pic:rolling_shutter}}
                        \label{pic:rolling_shutter}
                    \end{figure}

                    \begin{figure}[h]
                        \centering
                        \includegraphics[width=8cm]{ch2pic/rolling_shutter_case.png}
                        \caption{滾動式快門效應產生的條紋樣式\cite{pic:rolling_shutter_case}}
                        \label{pic:rolling_shutter_case}
                    \end{figure}

                    \qquad
                    光波段中除了環境光源會造成誤差以外,誤差主要來源於多重路徑傳輸,如圖\ref{pic:multipath}所示,訊號在傳送至感測器之前,由於障礙物造成訊號反射,使得感測器接收的訊號除了LoS路徑以外,還包含了透過反射傳送而來的訊號。除了LoS訊號以外,感測器量測到的NLoS訊號,便會造成誤差。而相機使用的為影像中的特徵點,因此較不受多重路徑傳輸的影響,在此相誤差上具有優勢。

                    \begin{figure}[h]
                        \centering
                        \includegraphics[width=6cm]{ch2pic/multipath.png}
                        \caption{多重路徑傳輸\cite{pic:multipath}}
                        \label{pic:multipath}
                    \end{figure}

                    \item[- Photodiode(PD)]\hfill
                    
                    \qquad
                    光電二極體(Photodiode,以下簡稱PD)能感測到資訊僅有強度資訊,而強度資訊受LED出射角度、PD入射角度、距離影響(詳細參考\ref{chp:LEDPD_radiate}章),相較大多無線電波技術主要僅受距離影響,PD的變數較多,使得獲得相對定位的演算法較複雜,也需要多個PD或多個LED以進行定位。然而PD優勢為具有極高的取樣頻率,可高達千赫茲甚至萬赫茲,高取樣頻率即可使用光通訊進行編碼與解碼,且硬體成本相較影像感測器來說非常低。

                    \qquad
                    PD定位系統的架構如圖\ref{pic:vlc_flow}所示,訊號發送是利用驅動器與LED進行光通訊,PD接收訊號後進行解碼、取得LED的ID與與光強度,再利用光強度以不同演算法解出相對位置。而LED與PD的定位系統中,包含單LED對多PD、多LED對多PD、或是多LED對多PD,其中,當系統內包含多個LED對多PD時,PD會同時接收到來自多個LED的編碼訊號,參考圖\ref{pic:vlc_multi_pd},PD會將訊號轉換至頻域,以獲得各頻率下的強度,也就是各LED傳送至各PD的強度。

    
                    
                    \begin{figure}[h]
                        \centering
                        \includegraphics[width=15cm]{ch2pic/vlc_multi_pd.png}
                        \caption{多PD進行光通訊解碼\cite{case:ml}}
                        \label{pic:vlc_multi_pd}
                    \end{figure}

                    \qquad
                    利用光通訊獲得各LED傳送至各PD的強度後,解出定位的演算法分成兩類,一類型為使用指紋法或多點定位,這種類型的系統需在環境中放置多個參考點(如圖\ref{pic:env_finger}),為「環境對一點的定位」。除此之外,大多使用\ref{chp:method-algorithm}章中提到的幾何方法,如\cite{case:3d_layers}(參考圖\ref{pic:env_1to1}),透過多PD量測強度後,利用多個PD強度與幾何之間的關係取得定位,此類型方法詳述於\ref{chp:LEDPD_problem}章。

                    \begin{figure}[h]
                        \centering
                        \includegraphics[width=8cm]{ch2pic/env_finger.png}
                        \caption{使用指紋法的系統環境\cite{case:ml}}
                        \label{pic:env_finger}
                    \end{figure}

                    \begin{figure}[h]
                        \centering
                        \includegraphics[width=8cm]{ch2pic/env_1to1.png}
                        \caption{使用幾何方法的系統環境\cite{case:3d_layers}}
                        \label{pic:env_1to1}
                    \end{figure}

                \end{description}

                影像感測器與PD進行定位的方法差異非常大,前者演算法為影像辨識,利用特殊圖案分辨光源,以影像中的變形計算位置;PD常見的定位演算法則包含多點定位、指紋法、幾何方法,利用光通訊分辨光源。在兩者之間權衡時,本研究主要考量到成本以及硬體大小,為了能夠達到靈活、廣泛運用,取捨掉影像感測器具有高精度可能性的優勢。


                \begin{figure}[h]
                    \centering
                    \includegraphics[width=9cm]{ch2pic/pd_camera_sort.png}
                    \caption{影像感測器與PD的特性比較}
                    \label{pic:pd_camera_sort}
                \end{figure}

                


        
                

        \subsection{小結:本研究使用技術的選擇}
        

        在\ref{chp:technique}章中,由電磁波頻率分類為無線電波段與光波段,分別於\ref{chp:radio}章與\ref{chp:light}章中介紹無線電波與光波段的定位,並於圖\ref{pic:method_compare}中比較兩者,其中光波段較高的精度、較小的硬體、以及能夠零或應用的幾何方法演算法,使光波段定位為更適合的選擇。

        光波段中,又依照使用波段與使用的硬體分類。使用波段於\ref{chp:light_electro}章中討論,分為可見光、近紅外光與遠紅外光,參考圖\ref{pic:light_freq_compare},近紅外光兼具低成本低干擾的特色,因此選用近紅外光波段中,位於輻射光譜低谷的760nm與940nm波段最為合適。硬體選擇則於\ref{chp:light_receiver}章中討論,參考圖\ref{pic:pd_camera_sort},權衡之下選擇PD,雖然捨棄精度,但得以換取較低的成本與較小的體積;另外,PD所使用的定位演算法需為具有靈活度的幾何方法。
    
        
        綜上所述,本論文將使用LED與PD的定位方法,於後續的\ref{chp:LEDandPD}章中更深入的討論,分別介紹LED與PD的特性、光傳遞模型,以及現今文獻所遇困難與限制。


        

    

        

        

        

        


        

        

            

          

        


\section{LED與PD的定位方法}
\label{chp:LEDandPD}
    


    
    於\ref{chp:technique}章中,我們將定位方法聚焦在LED與PD的定位方法上,而此種系統是將單個或多個LED安裝在目標物上,各LED藉由編碼來區別各自的訊號,而PD安裝於觀察物上,將每個PD所接收的訊號分別解碼,得到各LED的訊號強度。

    為深入了解LED與PD定位系統的運作方式,我們先於\ref{chp:LEDPD_flow}介紹LED與PD定位系統的流程,並於\ref{chp:light_unit}章介紹常用的光學單位;有基本了解後,於\ref{chp:LEDPD_radiate}介紹LED與PD的輻射特性,再於\ref{chp:model}章中介紹完整的光傳遞模型,最後再進入\ref{chp:LEDPD_problem}章介紹現今文獻的作法,並整理所遇困難與限制。

    \subsection{LED與PD定位架構與流程}
    \label{chp:LEDPD_flow}

    \begin{figure}[h]
        \centering
        \includegraphics[width=7cm]{ch2pic/lp_system_structure.png}
        \caption{LED與PD定位系統架構}
        \label{pic:lp_system_structure}
    \end{figure}

    LED與PD的定位系統,是由$L$個LED與$P$個PD組成,$L$與$P$的數量可為一也可為多個。根據\ref{chp:motivate}章中提到的使用情境,我們希望將LED與PD各自封裝成兩硬體單位,因此可用兩座標系表示兩硬體單位。其中,多個LED固定於LED座標系上,PD則是固定在PD座標系上,而兩座標系的相對關係可以自由的改變,參考\ref{chp:relative}章。實際的系統架構,可以參考圖\ref{pic:env_finger},其為\cite{case:ml}中的系統架構,而圖\ref{pic:ml_pd_config}則為該研究中的PD量測單位。

    \begin{figure}[h]
        \centering
        \includegraphics[width=10cm]{ch2pic/ml_pd_config.png}
        \caption{\cite{case:ml}研究的硬體組態設計}
        \label{pic:ml_pd_config}
    \end{figure}

    下圖則是系統的定位流程,為了能夠辨識是哪個LED發出的訊號,需要透過光通訊技術,由LED進行編碼,透過光傳遞到PD後。各個PD進行光通訊解碼,綜合所有資訊,我們可以得到一個L個LED對應P個PD的電流強度表格;在利用此表格進行定位的計算。


    \begin{figure}[h]
        \centering
        \includegraphics[width=15cm]{ch2pic/lp_system_flow.png}
        \caption{LED與PD定位系統流程}
        \label{pic:lp_system_flow}
    \end{figure}

    首先,我們來介紹現今LED與PD定位的系統流程:參考圖\ref{pic:lp_system_structure},前置作業需設計系統,包含挑選硬體與設計硬體組態。在這之中,LED與PD挑選時需注意的硬體特性分別於\ref{chp:LED}章與\ref{chp:PD}章中介紹,硬體組態的自由度則於\ref{chp:config}章中介紹。
    
    完成系統的設計與搭建後,系統運作的流程可以分為兩部分:光通訊與計算定位,以下條列介紹。

    

    \begin{description}
        \item[- 光通訊]\hfill
        
        \qquad 
        光通訊可參考圖\ref{pic:vlc_flow},各LED通過編碼,傳輸特定模式的訊號,在定位上最常使用的編碼系統為OOK,僅利用開關頻率將LED的ID傳輸。LED發出的光訊號透過光傳遞至PD,每個PD將所收到的訊號進行光通訊解碼,將訊號拆成訊號來自哪個LED以及該LED所造成的電流大小,綜合$P$個PD解碼後的資訊,即可獲得一$L\times P$的表格,顯示各PD對應各LED的電流。
        
        \item[- 定位計算] \hfill
        
        \qquad
        定位計算著重在如何透過$L\times P$各PD對應各LED的電流,回推出兩座標系的相對位置,如同\ref{chp:method-algorithm}章敘述,PD能使用的定位演算法種類包含多點定位、三角法、指紋法與幾何類型,前三種都為多點對一目標物的定位,系統環境與架構可參考圖\ref{pic:env_finger},並不符合本研究目標欲達到的兩單位之間的互相定位,參考圖\ref{pic:env_1to1};因此,以下針對幾何類型介紹。

        \qquad
        使用幾何方法的演算法,主要是仰賴光傳播模型中(式\ref{eqn:model}),PD的電流與距離、出入射角相關,利用多個LED與PD之間的交互關係,得到相對位置。光傳遞模型中,變數除了距離以外,還包含出入射角,使得PD比起僅能得到距離資訊的感測器,靈活度更高,例如PD得以透過特殊的限制與組態,得到目標物的姿態\cite{case:orient},是除了相機以外能得到姿態資訊的方法。然而也因為變數較多,使得模型較為複雜,如同式\ref{eqn:model_coor_extend}呈現,因此需透過適當的限制簡化模型複雜度,常見的限制於\ref{chp:LEDPD_restrict}章中介紹。
    \end{description}


    \subsection{光學常用單位介紹}
    \label{chp:light_unit}
        
        由於光領域所使用的單位與機械領域差距較大,且單位與術語經常有口語或混用的情況,令人感到困惑,因此在進入PD與LED定位的探討之前,先對光領域的一些術語與單位進行介紹,方便\ref{chp:lambertian}章以及\ref{chp:model}章中敘述朗博輻射模型與光傳播模型。

        首先,描述光照的單位分為兩種系統,輻射測量學(Radiometry)與光度測量學(Photometry),兩領域以不同單位描述光源,其中輻射測量學著重在電磁波輻射的量測,描述通量單位為瓦特(Watt);而光度測量學著重在人演可見之可見光波段的研究,通量單位為流明(Lumen),同樣物理量下單位定義不同\cite{radiometry_and_photometry}。
        

        \begin{figure}[ht]
            \centering
            \includegraphics[width=14cm]{ch2pic/photometry_table.png}
            \caption{輻射測量學與光度測量學的物理量比較}
            \label{tab:photometry}
        \end{figure}


        本研究聚焦在近紅外光波段,因此本論文會使用輻射測量學的系統,然而文獻上可見光定位數量較多,因此在單位的換算上需特別注意,以下針對光領域常用物理量分項簡單介紹。

        
        \begin{description}
            \item[- 立體角 Solid Angle $\Omega$] \hfill
                
                \qquad
                描述二維空間中的角度單位為弧度,代表夾角內的弧長與半徑比例,而單位弧度的定義是半徑與圓弧長度相等時的圓心角。

                \qquad
                然而光源存在於立體空間中,而空間中描述角度的物理量即為立體角$\Omega$(Solid Angle),該物理量代表一光束投影於單位球面上時,表面積與半徑平方的比例,如圖\ref{pic:solid_angle}所示。立體角的單位為球面度(steradians, 簡寫$sr$)代表在半徑為$r$的球體中,立體角投射出的表面積為$r^2$。

                \begin{figure}[h!]
                    \centering
                    \includegraphics[width=12cm]{ch2pic/solid_angle.png}
                    \caption{角度與立體角}
                    \label{pic:solid_angle}
                \end{figure}

            \item[- 輻射通量 Radiant Flux $\Phi$]  \hfill
                
                \qquad
                描述光照功率的物理量為通量(Flux)或稱輻射功率(Power),用符號$\Phi$表示,代表每單位時間的輻射能量,單位為瓦特,而大多LED規格表上以此物理量來描述LED在指定電流下可產生的最大光功率。

            \item[- 輻射強度 Radiation Intensity $I$] \hfill
             
                \qquad
                每單位立體角所含的通量稱為輻射強度$I$,此物理量常用於描述光與立體角之關係,在同一光束內,輻射強度僅與立體角有關,與距離無關。

                \begin{figure}[h!]
                    \centering
                    \includegraphics[width=8cm]{ch2pic/intensity_irradiance.png}
                    \caption{輻射強度與輻照度}
                    \label{pic:intensity_irradiance}
                \end{figure}

            \item[- 輻照度 Irradiance $E$] \hfill
                
                \qquad
                每單位面積所含的通量稱為輻照度(參考圖\ref{pic:intensity_irradiance}),其中照射面積隨著距離$D$增加而平方遞增。

                
                % \begin{equation}
                %     \label{eqn:I2E}
                %     E=I\cos\omega/D^2
                % \end{equation}      
             
        \end{description}

        

        

    \subsection{LED與PD的輻射特性}
    \label{chp:LEDPD_radiate}

    LED與PD硬體有許多種類,最常見的市面上LED與PD為軸對稱並滿足朗博輻射模式(Lambertian Radiation Pattern),本論文僅考慮此類硬體,其中軸對稱代表輻射強度並不隨方位角改變。

    以下章節依序從\ref{chp:config}章中,定義軸對稱的LED與PD組態,介紹硬體組態的自由度,於\ref{chp:lambertian}章介紹LED與PD皆遵守的朗博輻射模型(Lambertian Radiation Pattern),再於\ref{chp:LED}與\ref{chp:PD}章分別介紹LED與PD兩者各自的硬體特性。
        
        \subsubsection{LED與PD的組態}
        \label{chp:config}


        \begin{figure}[ht]
            \centering
            \includegraphics[width=12cm]{ch2pic/vec_config.png}
            \caption{三維空間中的向量}
            \label{pic:vec_config}
        \end{figure}

        
        在描述三維空間中的LED與PD時,需要描述硬體的擺放位置,而由於LED與PD皆有中心軸,因此需要用向量表示指向。如圖\ref{pic:vec_config},圓點代表硬體位置,而向量表示硬體指向,其中位置以$\boldsymbol{P}$表示,包含三個自由度$[x,y,z]^T$,指相則由$\boldsymbol{V}$表示,自由度包含$[u,v,w]^T$,而由於LED與PD指向只在乎方位不在乎距離,因此可將指向向量定義為單位向量:$||\boldsymbol{P}||=1$。指向也常以球座標系表示$\boldsymbol{V}^{sph}$,三個分量為$[d,\alpha,\beta]^T$,其中指向的距離分量$d$為一,而球座標系與卡氏座標系之間的換算可參考式\ref{eqn:car2sph}。
        
        以LED為例(參考圖\ref{pic:led_config}),我們需要定義LED硬體於LED座標系中的位置$^{L}\boldsymbol{P}$與指向$^{L}\boldsymbol{V}$,,左上標$L$代表其定義於LED座標系中。一個LED的組態一共有五個自由度,包含擺放位置$^L \boldsymbol{P}$的三個自由度$[^Lx,^Ly,^Lz]^T$,以及定義指向的單位向量$^L\boldsymbol{V}$所含兩個自由度:天頂角$^L\alpha$與方位角$^L\beta$,換算至卡氏座標系則為$[^Lu,^Lv,^Lw]^T$。

        \begin{figure}[ht]
            \centering
            \includegraphics[width=10cm]{ch2pic/LED_config.png}
            \caption{LED在座標系中的擺放自由度}
            \label{pic:led_config}
        \end{figure}


       而PD的擺放自由度相同,唯一差別為定義的座標系為PD座標系,因此左上標為$P$,如擺放位置為$^P \boldsymbol{P}=[^Px,^Py,^Pz]^T$,定義指向的單位向量$^P\boldsymbol{V}^{sph}=[1,^P\alpha,^P\beta]^T$。
        




        \subsubsection{朗博輻射模型}
        \label{chp:lambertian}

        \begin{figure}[h]
            \centering
            \includegraphics[width=8cm]{ch2pic/lambertian.png}
            \caption{輻射強度與出入射角的關係}
            \label{pic:lambertian}
        \end{figure}

        LED與PD的照射與接收模式(Pattern)皆可以用朗博輻射模式描述(圖\ref{pic:lambertian}),其代表感(發)光強度隨著LED出射角(PD入射角)的增加而衰減,如式\ref{eqn:lambertian_pattern}描述,感(發)光強度於中心軸時最大,其衰減模式可用餘弦函數(cosine)的$M$次方(power)表示,$M$代表的是朗博次方(Lambertian Order),需注意的是輻射強度與距離無關(參考\ref{chp:light_unit}章)。

        \begin{equation}
            \label{eqn:lambertian_pattern}
            I(\omega)=I(\omega=0)\times\cos(\omega)^{M}\\
        \end{equation}

     
        上述為輻射強度與出入射角的關係,為了呈現出入射角於空間中的關係,我們以一向量描述LED與PD的中心軸方位,而由於LED與PD最大的可視範圍不會超過出射角$90^{o}$,因此用半球體描述LED與PD的可視範圍,呈現於圖\ref{pic:angle_3d}。如圖所示,出入射角在空間中僅與球座標系中的天頂角$\alpha$相關,在同樣天頂角時出入射角皆相同。

        以圖\ref{pic:angle_3d}作延伸,空間中的LED與PD,朗博輻射模式在朗博次方為一時如圖\ref{pic:lambertian_3d},在同樣出射角下,光輻射強度相同,呈現軸對稱,愈接近中心軸的輻射強度愈大。

        \begin{figure}[ht]
            \centering
            \includegraphics[width=8cm]{ch2pic/3d_angle.png}
            \caption{LED與PD於三維空間中的入射角}
            \label{pic:angle_3d}
        \end{figure}

        \begin{figure}[ht]
            \centering
            \includegraphics[width=12cm]{ch2pic/lambertian_3d.png}
            \caption{朗博次方為一時空間中輻射強度與出入射角方位的關係}
            \label{pic:lambertian_3d}
        \end{figure}

        而朗博次方所代表的意義,可視為在敏感度與照射範圍中的取捨:朗博次方較小時,其照射範圍較大,最大可以覆蓋接近半個球面,光強度隨出射角度衰減的速度不大,也就是對不同入射角度的敏感度則不高。反之,朗博次方較大的硬體雖然覆蓋範圍較小,但對覆蓋範圍內的角度變化敏感度高(如圖\ref{pic:lambertian})。

        

        \subsubsection{LED特性}
        \label{chp:LED}

        LED與PD在選購時,除了挑選朗博次方以外,也需注意其他特性以及影響發收光能力的硬體參數,在此將硬體參數整理於圖\ref{pic:hardware_para},這些硬體參數會陸續於\ref{chp:LED}章與\ref{chp:PD}章中介紹。
        
        \begin{figure}[ht]
            \centering
            \includegraphics[width=9cm]{ch2pic/hardware_para.png}
            \caption{LED與PD的硬體參數}
            \label{pic:hardware_para}
        \end{figure}

        LED發出的輻射強度$I\ell$,會遵守式\ref{eqn:lambertian_pattern}描述的朗博輻射模式,其提供了LED光輻射強度$I\ell$與不同出射角$\theta$的關係。而由於描述LED特性時,多以總輻射通量$Pt$描述,而總輻射通量為半球面空間中的輻射強度$I\ell$積分,如式\ref{eqn:lambertian_integral}。

        \begin{equation}
            \label{eqn:lambertian_integral}
            Pt = I\ell(0)(M\ell+1)/2\pi
        \end{equation}
        
        透過式\ref{eqn:lambertian_integral}移項與式\ref{eqn:lambertian_pattern},即可完成式\ref{eqn:lambertian_led},描述LED輻射強度$I\ell$與出射角度$\theta$、LED總輻射功率$Pt$以及朗博次方$M\ell$的關係。
    
        \begin{equation}
            \label{eqn:lambertian_led}
            I\ell(\theta)=Pt\frac{(M\ell+1)}{2 \pi} \cos \theta^{M\ell}
        \end{equation}

        圖\ref{pic:lambertian_led}呈現了LED在總輻射通量$Pt$相同的情況下,輻射強度$I\ell$與出射角$\theta$的關係。圖中呈現了式\ref{eqn:lambertian_integral}積分的影響,在朗博次方較大時,中心軸的輻射強度比值突出,其原因為朗博輻射模式於半球面上積分的影響。擁有固定輻射總通量時,我們將輻射強度分配給小的入射角度,來分總通量的立體角不多;反之當我們將輻射強度分配給大的入射角度時,入射角大所包含的立體角較多。因此,朗博次方較小時,隨著出入射角度增加,輻射強度衰減速度緩慢,出射角於45度仍有中心軸0.7071倍的強度,因此總通量需分配不少於大角度的部分;而朗博次方較大時,大出射角的輻射強度極小,僅用將總輻射通量分配給小出射角的部分,因此輻射強度對總輻射強度的比值較大。

        \begin{figure}[h]
            \centering
            \includegraphics[width=9cm]{ch2pic/lambertian_led.png}
            \caption{輻射強度與總輻射通量比值}
            \label{pic:lambertian_led}
        \end{figure}



        \subsubsection{PD特性}
        \label{chp:PD}

        首先,PD得到的輻射強度$Ip$也遵守朗博輻射模式,式\ref{eqn:pd_I}描述輻射強度與入射角度$\phi$以及PD朗博次方$Mp$的關係。

        \begin{equation}
            \label{eqn:pd_I}
            Ip(\phi) = Il \cos \phi^{Mp}
        \end{equation}

        而PD所量測到的為輻射通量$\Phi_p$,再將輻射通量轉換為電流輸出(式\ref{eqn:pd_current})。PD量測到的輻射通量呈現於式\ref{eqn:pd_flux},藉由圖\ref{pic:intensity_irradiance}中輻射通量$\Phi$與輻射強度$I$的關係以及立體角$\Omega$的定義,將式\ref{eqn:pd_I}的光強度轉換為PD接收的輻射通量$\Phi p$。其中,$A$為PD的硬體參有效面積,$D$為LED與PD之間的距離。

        \begin{equation}
            \label{eqn:pd_flux}
            \Phi p = \frac{Ip}{\Omega} = \frac{IpA}{D^2}
        \end{equation}

        而PD最終輸出電流$Ie$與輻射通量$\Phi p$的關係由式\ref{eqn:pd_current}描述,$s$為PD的飽和電流,也就是該PD能夠輸出的最大電流:在未達飽和之前,輸出電流$Ie$與輻射通量$\Phi p$成正比,比值為響應率$Re$。

        \begin{equation}
            \label{eqn:pd_current}
            Ie = \begin{cases}\Phi\times Re, & \text { if } Ie<s \\ s, & \text { otherwise }\end{cases}
        \end{equation}

        綜合式\ref{eqn:pd_I}、式\ref{eqn:pd_flux}、式\ref{eqn:pd_current},PD輸出的電流$Ie$與LED的輻射強度$I\ell$統整於式\ref{eqn:pd},式中的硬體參數可參考圖\ref{pic:hardware_para}:

        \begin{equation}
            \label{eqn:pd}
            Ie = \begin{cases}Re \frac{A}{D^2}Il(\theta) \cos \phi^{Mp}, & \text { if } Ie<s \\ s, & \text { otherwise }\end{cases}
        \end{equation}

        
        


        

    \subsection{光傳遞模型}
    \label{chp:model}

    本章節將綜合\ref{chp:LED}章與\ref{chp:PD}章中呈現的LED與PD模型,於\ref{chp:model_1to1}章中,呈現由LED光源將光波傳送至PD並轉換成電流的完整模型;接著於\ref{chp:model_mul}章中,將LED與PD的數量提升為多個;最後於\ref{chp:model_transform}章中將光傳播模型中的距離$D$、出入射角$\theta\phi$變數轉換為用LED與PD的座標系表示,呈現出LED與PD在不限制條件的情況下模型非常複雜。
    
    \subsubsection{單LED對單PD光傳遞模型}
    \label{chp:model_1to1}

    光傳遞模型需建立由LED至PD的過程,LED與PD各自的輻射與接收特性分別於\ref{chp:LED}章與\ref{chp:PD}章中介紹,綜合描述LED發光強度$Il$的式\ref{eqn:lambertian_led},以及描述PD輸出電流$Ie$的式\ref{eqn:pd},即可獲得一LED對一PD的光傳遞模型(式\ref{eqn:model_1to1})。

    \begin{equation}
        \label{eqn:model_1to1}
        Ie = \begin{cases}Re \cdot A\cdot Pt\frac{M\ell+1}{2\pi}\frac{\cos \phi^{Mp}\cos \theta^{M\ell}}{D^2}, & \text { if } Ie<s \\ s, & \text { otherwise }\end{cases}
    \end{equation}

    在光傳遞模型中,與定位相關的變數包含三個:距離$D$、PD入射角$\phi$、LED出射角$\theta$,他們便是幾何方法演算法要使用的變數。其中出入射角如圖\ref{pic:interactive_1to1},分別為硬體中心軸與距離向量之間的夾角。

    \begin{figure}[h]
        \centering
        \includegraphics[width=6cm]{ch2pic/interactive_1to1.png}
        \caption{單LED與單PD的交互關係}
        \label{pic:interactive_1to1}
    \end{figure}

    \subsubsection{多LED對多PD光傳遞模型}
    \label{chp:model_mul}

    由於LED與PD定位時數量通常不只一個,因此描述多LED對多PD的光傳遞模型時,需註明清楚(可參考第\pageref{chp:symbol}頁)。在此設定LED總數為$L$、PD總數為$P$,描述第$l$個LED與第$p$個PD的硬體參數時,於硬體參數的右下標加註,如$A_2$為第二個PD的有效面積;再來,描述第$l$個LED與第$p$個PD之間的關係時,則將LED與PD的編號$lp$同樣加註於右下標,例如$D_{34}$為第三個LED與第四個PD之間的距離,參考圖\ref{pic:interactive_mul}。
    
    \begin{figure}[h]
        \centering
        \includegraphics[width=8cm]{ch2pic/interactive_mul.png}
        \caption{多LED與多PD的交互關係}
        \label{pic:interactive_mul}
    \end{figure}

    而完整的多LED對多PD的的模型如式\ref{eqn:model},$Ie_{lp}$代表著第$p$個PD收到來自第$l$個LED所產生的電流大小。
    
    \begin{equation}
        \label{eqn:model}
        Ie_{lp} = \begin{cases}Re_p \frac{ A_p\cos\phi_{lp}^{Mp_{p}} }{D^2_{lp}}\times Pt_l\frac{(M\ell_{l}+1)}{2 \pi} \cos \theta_{lp}^{M\ell_{l}}, & \text { if } Ie_{lp}<s_p \\ s_p, & \text { otherwise }\end{cases}
    \end{equation}

    




    \subsubsection{以座標系呈現光傳遞模型}
    \label{chp:model_transform}

    雖然式\ref{eqn:model}已呈現了多LED對多PD的光傳遞模型,然而式子中的變數為距離$D$與出入射角$\theta,\phi$,而這些變數為各LED與各PD之間的幾何交互關係,並沒有將LED與PD定義於各自的座標系中,參考圖\ref{pic:relative_vs_coor},呈現\ref{chp:relative}章中描述的兩座標系各自移動的情境。我們需要得到「相對位置與PD輸出電流大小」的關係,才能夠利用電流關係得到相對位置關係,因此需要將式\ref{eqn:model}中的幾何交互關係,改以兩座標系轉換關係與組態來表示。

    \begin{figure}[h]
        \centering
        \includegraphics[width=8cm]{ch2pic/relative_vs_coor.png}
        \caption{LED與PD之間的關係以座標系之間的關係表示}
        \label{pic:relative_vs_coor}
    \end{figure}

    為了達到此目的,首先,如同\ref{chp:config}章中所描述,LED與PD的組態各自定義於LED與PD座標系中,需定義的包含位置$\boldsymbol{P}$與指向$\boldsymbol{V}$。\ref{chp:relative}章中則定義了兩座標系之間的齊次座標轉換矩陣為$^{PL}\boldsymbol{H}$,左上標的$PL$代表將LED座標系投影至PD座標系上,而要得到的相對位置資訊即為平移矩陣$^{PL}\boldsymbol{T}$。

    我們為了要計算出兩座標系上硬體的幾何關係,需將LED座標系上的所有LED位置與指向,透過齊次座標轉換矩陣$^{PL}\boldsymbol{H}$,投影到PD座標系,再於PD座標系中計算交互關係,流程圖如圖\ref{pic:model_transform_flow}。
    

    \begin{figure}[h]
        \centering
        \includegraphics[width=15cm]{ch2pic/model_transform_flow.png}
        \caption{光傳遞模型與相對定位統整的流程}
        \label{pic:model_transform_flow}
    \end{figure}
    
    \begin{description}

        

        \item[- 將LED投影到PD座標系上]\hfill 
        
        \qquad
        LED由$^L \boldsymbol{P}$代表LED位置、$^L \boldsymbol{V}$代表朝向,於式\ref{eqn:homogeneous_point}呈現出$^L \boldsymbol{P}$經過齊次座標轉換至PD座標系上$^{PL} \boldsymbol{P}$。而$^L \boldsymbol{V}$為單位項量,由於朝向不需經過平移,可以僅用旋轉矩陣$\boldsymbol{Ro}$換算,由式\ref{eqn:homogeneous_vector}轉換至PD座標系上為$^{PL}\boldsymbol{V}$。

        \begin{equation}
            \label{eqn:homogeneous_point}
            \begin{aligned}
            \left[\begin{array}{cc}
            { }^{P L} \boldsymbol{P} \\
            1
            \end{array}\right]&={ }^{P L} \boldsymbol{H}\left[\begin{array}{c}
            { }^{L} \boldsymbol{P} \\
            1
            \end{array}\right] &=\left[\begin{array}{cc}
            { }^{P L} \boldsymbol{R} \boldsymbol{o} & { }^{P L} \boldsymbol{T} \\
            0 & 1
            \end{array}\right]\left[\begin{array}{c}
            { }^{L} \boldsymbol{P} \\
            1
            \end{array}\right] \\
        \end{aligned}
        \end{equation}
        
        \begin{equation}
            \label{eqn:homogeneous_vector}
            \begin{aligned}
            ^{P L} \boldsymbol{V} &=^{P L} \boldsymbol{R} \boldsymbol{o} ^{L}\boldsymbol{ V} 
            \end{aligned}
        \end{equation}




        \item[- 於PD座標系中計算距離、出入射角]\hfill 
        
        \qquad
        首先,我們在PD座標系中計算第$l$個LED與第$p$個PD之間的距離,距離大小以$D_{lp}$表示,而PD指向LED位置的距離向量則以粗體表示:$\boldsymbol{D}_{lp}$,於式\ref{eqn:distance_to_coor}呈現。
        
        \begin{equation}
            \label{eqn:distance_to_coor}
            \begin{aligned}
            \boldsymbol{D}_{lp} = ^{PL}\boldsymbol{P}_l- ^{P}\boldsymbol{P}_p \\
            D_{lp} = ||\boldsymbol{D}_{lp}||
            \end{aligned}
        \end{equation}

        \qquad
        再來,第$l$個LED對第$p$個PD出射角$\theta_{lp}$與入射角$\phi_{lp}$,可透過餘弦與內積的關係求得,於式\ref{eqn:angle_to_coor}表示。

        \begin{equation}
            \label{eqn:angle_to_coor}
            \begin{aligned}
            \theta_{lp} = \arccos(-\frac{^{PL}\boldsymbol{V}_l \cdot {\boldsymbol{D}_{lp}}}{D_{lp}})\\
            \phi_{lp} = \arccos(\frac{^{P}\boldsymbol{V}_p \cdot {\boldsymbol{D}_{lp}}}{D_{lp}})\\
            \end{aligned}
        \end{equation}

        \qquad
        將距離、出入射角都以PD座標系表示後,於式\ref{eqn:model_coor}將整個光傳播模型(式\ref{eqn:model})以組態與座標系轉換表示:

        \begin{equation}
            \label{eqn:model_coor}
            \begin{aligned}
                \text { if } Ie_{lp}<s_p \\
                Ie_{lp} &= \frac{Re_pA_pPt_l(M\ell_{l}+1)}{2 \pi} \frac{ {\cos\phi_{lp}}^{Mp_{p}} \times{\cos \theta_{lp}}^{M\ell_{l}}}{D^2_{lp}}\\
                    & = \frac{Re_pA_pPt_l(M\ell_{l}+1)}{2 \pi} \frac{ {{(^{P}\boldsymbol{V}_p \cdot {\boldsymbol{D}_{lp}})}}^{Mp_{p}} {(-^{PL}\boldsymbol{V}_l \cdot {\boldsymbol{D}_{lp}})}^{M\ell_{l}}}  {{||{\boldsymbol{D}_{lp}}||}^{2+M\ell_l+Mp_p}}
            \end{aligned}
        \end{equation}

        \qquad
        其中,${\boldsymbol{D}_{lp}}$、$||{\boldsymbol{D}_{lp}}||$、$^{PL}\boldsymbol{V}_l$都還未完全展開,我們將其完全展開呈現於式\ref{eqn:model_coor_extend},進而得出結論:無論是硬體擺設方式$\boldsymbol{P}$或$\boldsymbol{V}$更換,或是兩座標系相對關係$^{PL}\boldsymbol{H}$改變,皆會影響到輸出電流,是一個牽一髮動全身的模型,我們難以輕易透過輸出電流,獲得相對位置$^{PL}\boldsymbol{T}$。因此,所有的研究皆需利用不同的限制方式,使式\ref{eqn:model_coor}簡化,以取得相對位置資訊。


        \begin{equation}
            \label{eqn:model_coor_extend}
            \begin{aligned}
                &\text { if } Ie_{lp}<s_p \\
                &Ie_{lp} \\&= \frac{Re_pA_pPt_l(M\ell_{l}+1)}{2 \pi} 
                \\
                &\qquad \times
                 \frac{ 
                    {(
                        ^{P}\boldsymbol{V}_p 
                        \cdot 
                        (
                            ^{PL}\boldsymbol{P}_l- ^{P}\boldsymbol{P}_p
                        )
                    )}
                    ^{Mp_{p}}
                    \times 
                    {(
                        -^{PL}\boldsymbol{Ro}^{L}\boldsymbol{V}_l 
                        \cdot 
                        ({
                            ^{PL}\boldsymbol{P}_l
                            - ^{P}\boldsymbol{P}_p
                        })
                    )}^{M\ell_{l}}
                } 
                  {{||^{PL}\boldsymbol{P}_l- ^{P}\boldsymbol{P}_p||}^{2+M\ell_l+Mp_p}}\\
                  &= \frac{Re_pA_pPt_l(M\ell_{l}+1)}{2 \pi}\\
                  &\qquad\times 
                  {( ^{P}\boldsymbol{V}_p \cdot 
                            (
                                (
                                    ^{PL} \boldsymbol{Ro}^{P}\boldsymbol{P}_l
                                    + ^{PL}\boldsymbol{T}
                                )
                                - ^{P}\boldsymbol{P}_p
                            )
                        )}^{Mp_{p}}\\
                &\qquad\times
                {
                            (
                                -^{PL}\boldsymbol{Ro}^{L}\boldsymbol{V}_l 
                                \cdot 
                                (
                                    (
                                        ^{PL}\boldsymbol{Ro}^{P}\boldsymbol{P}_l
                                        +^{PL}\boldsymbol{T}
                                    )
                                    - ^{P}\boldsymbol{P}_p
                                )
                            )
                        }^{M\ell_{l}}   \\
                & \qquad \times
                   \frac{     
                        1   
                    } 
                    {
                        {
                            ||
                                (^{PL}\boldsymbol{Ro}^{P}\boldsymbol{P}_l+^{PL}\boldsymbol{T})
                                - ^{P}\boldsymbol{P}_p
                            ||
                        }^{2+M\ell_l+Mp_p}
                    }\\
            \end{aligned}
        \end{equation}

        \end{description}   


        






    \subsection{LED與PD定位系統現況}
    \label{chp:LEDPD_now}
        
        簡單統整一下前面敘述的LED與PD定位系統資訊:首先,LED與PD皆為軸對稱,定義一空間中的LED或PD共有五個自由度(參考\ref{chp:config}章),且LED與PD皆遵守朗博輻射模式,藉由挑選朗博次方,可對出入射角敏感度與輻射範圍之間取捨(參考\ref{chp:lambertian}章)。接著,\ref{chp:LED}章中介紹了LED的特性,而\ref{chp:PD}章中介紹了PD接收光輻射的特性,綜合便可獲得\ref{chp:model}章中完整的光傳遞模型,並於\ref{chp:model_transform}章將光傳遞模型,以LED與PD座標系的座標轉換呈現。

        由於\ref{chp:model_transform}章中呈現的光傳遞模型,在完整考慮硬體組態的自由度、硬體種類的選擇等,會使模型非常複雜(如式\ref{eqn:model_coor_extend}),難以從中獲得PD電流強度與相對位置的關係,因此文獻上透過適當限制自由度以簡化模型,而常見限制於\ref{chp:LEDPD_restrict}章中介紹,而在\ref{chp:LEDPD_case}章中介紹幾個與本研究較相近的文獻,最後於\ref{chp:LEDPD_problem}章中統整現今LED與PD定位文獻上所不足的地方。



        

        


        \subsubsection{常見的LED與PD定位系統限制}
        \label{chp:LEDPD_restrict}

        現今文獻需加諸系統限制以求得定位,而常見的限制以下條列呈現:

        \begin{description}
            \item[$\cdot$ 忽略朗博次方] \hfill
            
            \qquad
            朗博次方是一個影響輻射模式的重要參數(參考\ref{chp:lambertian}章),LED與PD兩者皆有各自的朗博次方,代表著訊號對角度敏感度與覆蓋範位的取捨。大多文獻將LED與PD的朗博次方皆假設為一,降低系統複雜度,使式\ref{eqn:model_coor_extend}簡化於式\ref{eqn:model_no_lamb}。

            \qquad
            然而,假設朗博次方為一在實際上是不符合現實的,因為實際上在挑選硬體時朗博次方時有許多選擇,如\cite{datasheet:led_vsma}的朗博次方便為約5.57。為了降低複雜度而省略朗博次方,會侷限了能夠使用的硬體,也限制了系統的自由度\cite{survey_light2018}。即便如此,大多研究仍將其省略,其中將PD的朗博次方納入考量的文獻極少\cite{survey_light2018},相較起來,LED朗博次方有部分研究有將其納入計算,如\cite{case:cart2d}\cite{case:cart3d}。
            
            

            \item[$\cdot$ 硬體種類限制]\hfill
            
            \qquad
            文獻上,挑選LED時$L$個LED皆會選擇同一種,同一系統下,並不會同時出現多種LED,PD亦然。因此,LED與PD的硬體參數(參考圖\ref{pic:hardware_para})並不會隨LED與PD的編號改變,綜合此限制與朗博次方為一的限制,將式\ref{eqn:model_coor_extend}簡化為式\ref{eqn:model_no_lamb}。

            \begin{equation}
                \label{eqn:model_no_lamb}
                \begin{aligned}
                    &\text{Let }
                    \begin{cases}
                        Mp_p=Mp\\M\ell_l=M\ell\\Re_p=Re\\A_p=A\\Pt_l=Pt\\s_p=s
                    \end{cases}
                    \text{ for}
                    \begin{cases}
                        p = 1,2,...,P\\
                        l = 1,2,...,L\\
                    \end{cases}\\
                    &\text { When } Ie_{lp}<s ,\\
                    &Ie_{lp} \\
                        &= \frac{ReAPt}{ \pi}
                        {( ^{P}\boldsymbol{V}_p \cdot 
                                (
                                    (
                                        ^{PL} \boldsymbol{Ro}^{P}\boldsymbol{P}_l
                                        + ^{PL}\boldsymbol{T}
                                    )
                                    - ^{P}\boldsymbol{P}_p
                                )
                            )}\\
                    & \qquad \times
                        \frac{  
                            (
                                -^{PL}\boldsymbol{Ro}^{L}\boldsymbol{V}_l 
                                \cdot 
                                (
                                    (
                                        ^{PL}\boldsymbol{Ro}^{P}\boldsymbol{P}_l
                                        +^{PL}\boldsymbol{T}
                                    )
                                    - ^{P}\boldsymbol{P}_p
                                )
                            )
                        } 
                        {
                            {
                                ||
                                    (^{PL}\boldsymbol{Ro}^{P}\boldsymbol{P}_l+^{PL}\boldsymbol{T})
                                    - ^{P}\boldsymbol{P}_p
                                ||
                            }^{4}
                        }\\
                \end{aligned}
            \end{equation}

            \item[$\cdot$ 限制PD組態]\hfill
            
            \qquad
            最常見的硬體組態限制,為限制所有PD的擺放位置於同一點:$^P\boldsymbol{P}_p=
            \left[\begin{array}{ccc}0&0&0\end{array}\right]^T$,則組態僅剩下朝向$^P\boldsymbol{V}_p^{sph} = \left[\begin{array}{ccc}1&\alpha&\beta\end{array}\right]^T$的兩個自由度,擺法可參考圖\ref{pic:config_orient}。現今幾乎所有幾何方法的PD組態都是此種類型\cite{survey_light2018}。通過此限制,式\ref{eqn:model_no_lamb}可進一步簡化為式\ref{eqn:model_config_restrict}。
            
            \begin{figure}[h]
                \centering
                \includegraphics[width=8cm]{ch2pic/config_orient.png}
                \caption{常見限制擺放於同點位置PD組態示意圖\cite{case:3d_layers}}
                \label{pic:config_orient}
            \end{figure}

            \begin{equation}
                \label{eqn:model_config_restrict}
                \begin{aligned}
                    &\text{Let }
                    ^P\boldsymbol{P}_p=
                    \left[\begin{array}{ccc}0&0&0\end{array}\right]^T
                    \text{ for } p = 1,2,...,P\\
                    &\text { If } Ie_{lp}<s ,\text{ then }Ie_{lp} \\
                        &= \frac{ReAPt}{ \pi}
                    \frac{
                            {( ^{P}\boldsymbol{V}_p \cdot 
                                (
                                    ^{PL} \boldsymbol{Ro}^{P}\boldsymbol{P}_l
                                    + ^{PL}\boldsymbol{T}
                                )
                            )}
                    {
                                (
                                    -^{PL}\boldsymbol{Ro}^{L}\boldsymbol{V}_l 
                                    \cdot 
                                    (
                                        ^{PL}\boldsymbol{Ro}^{P}\boldsymbol{P}_l
                                        +^{PL}\boldsymbol{T}
                                    )
                                )
                            } } 
                        {
                            {
                                ||
                                    ^{PL}\boldsymbol{Ro}^{P}\boldsymbol{P}_l+^{PL}\boldsymbol{T}
                                ||
                            }^{4}
                        }\\
                \end{aligned}
            \end{equation}

            \qquad
            除了位置的限制,PD指向在文獻中也常將兩自由度降至一個自由度,也就是將$^P\boldsymbol{V}_p^{sph}$的方位角平均分配:$\beta_p = 1\pi/P$,而各PD的天頂角則限制為相同:$\alpha_p = \alpha$,因此PD唯一的組態變數指剩下$\alpha$天頂角,如\cite{case:cart2d}\cite{case:cart3d}\cite{case:3d_layers}。另外一種限制,則是利用多面體,將PD朝向限制於多面體的頂點,如\cite{case:ml}。

            \item[$\cdot$ 限制定位維度]\hfill
            
            \qquad 
            現今使用幾何方法的LED與PD系統,大多定位維度僅到二維\cite{survey_light2018},將欲解的相對位置$^{PL}\boldsymbol{T}=\left[\begin{array}{ccc}x&y&z\end{array}\right]^T$中的$z$限制為已知,也就是將目標物與觀察者的垂直距離固定,參考圖\ref{pic:case_parallel}。由於大多LED與PD的定位系統應用室內原有的光源,因此此限制代表的意義,便是目標物僅能於室內地面的水平高度移動,無法定位不在地面上的目標物。透過此限制,在方程式上的意義,則是將變數由三個減少為兩個。

            \begin{figure}[h]
                \centering
                \includegraphics[width=8cm]{ch2pic/case_parallel.png}
                \caption{常見的LED與PD定位系統架設\cite{case:aoa}}
                \label{pic:case_parallel}
            \end{figure}
            
            \qquad
            少數達到三維定位的LED與PD定位系統,大多是利用指紋法\cite{case:ml},然而指紋法為環境對單點的定位,不符合本研究需求。而利用幾何方法達到三維定位的研究較少,\cite{case:cart3d}透過一已知$x,y$位置的參考點,來進行垂直距離的計算;\cite{case:3d_layers}則是使用複雜的演算法,獲得目標物方位後,利用迭代計算垂直距離;\cite{case:hypercube}則需使用兩平行且之間有距離的LED,利用PD兩兩比較電流大小,計算出三維相對位置;以上三篇研究皆有限制使用情境,而這些研究於\ref{chp:LEDPD_case}章詳述。



            \item[$\cdot$ 限制使用情境]   \hfill
            
            \qquad
            在使用幾何類型演算法的研究中,最常見的使用情境限制為:目標平面與量測者平面需平行,如圖\ref{pic:case_parallel}。舉例來說,展場內的光源中心軸朝地面,此時利用手機前鏡頭進行定位的使用者,便需將手機平放,以使其中心軸垂直地面朝上達到平行。

            \qquad
            透過此限制,光傳遞模型中的出射角$\theta$必與入射角$\phi$相等,即可將光傳播模型的變數降低至兩個,利用$\cos\theta=\cos\phi$可將式\ref{eqn:model_config_restrict}簡化至式\ref{eqn:model_parallel_restrict},其中的$^{PL}\boldsymbol{Ro}$也僅剩一個自由度,大大的簡化了模型複雜度。


            \begin{equation}
                \label{eqn:model_parallel_restrict}
                \begin{aligned}
                    &\text { If } Ie_{lp}<s ,\text{ then }Ie_{lp} \\
                        &= \frac{ReAPt}{ \pi}
                    \frac{
                            {( ^{P}\boldsymbol{V}_p \cdot 
                                (
                                    ^{PL} \boldsymbol{Ro}^{P}\boldsymbol{P}_l
                                    + ^{PL}\boldsymbol{T}
                                )
                            )}^2
                     } 
                        {
                            {
                                ||
                                    ^{PL}\boldsymbol{Ro}^{P}\boldsymbol{P}_l+^{PL}\boldsymbol{T}
                                ||
                            }^{4}
                        }\\
                \end{aligned}
            \end{equation}

            \qquad
            雖然此限制常見,但是也造成很大的使用不便,除了室內光源搭配於地面上不改變垂直距離的移動載具以外,基本上沒有其他應用場域,大幅度限制了使用的情境,難以達到具有靈活度的應用,因此本研究的主要目標之一便是去除此限制。
     

        \end{description}
        


    \subsubsection{LED與PD定位系統實例}
    \label{chp:LEDPD_case}

    \begin{description}
        \item[\cite{case:hypercube}:使用紅外光且易於攜帶但量測範圍小的案例]\hfill 
        
        \qquad
        \cite{case:hypercube}研究利用2個LED與3個PD進行光通訊,光通訊流程如圖\ref{pic:vlc_flow},其將朗博次方假設為一,使用與圖\ref{pic:config_orient}相同的PD組態,實體圖為圖\ref{pic:hypercube_pd},為達到三維定位而限制目標物與量測者需為平行。

        \begin{figure}[h]
            \centering
            \includegraphics[width=6cm]{ch2pic/hypercube_pd.png}
            \caption{\cite{case:hypercube}的PD組態}
            \label{pic:hypercube_pd}
        \end{figure}
        \begin{figure}[h]
            \centering
            \includegraphics[width=9cm]{ch2pic/hypercube_setup.png}
            \caption{\cite{case:hypercube}的系統架構}
            \label{pic:hypercube_setup}
        \end{figure}
        


        \qquad
        簡單介紹此研究使用的演算法:首先,三個PD指向分別為$^P\boldsymbol{V}_1=[1,0,0]^T,^P\boldsymbol{V}_2=[0,1,0]^T,^P\boldsymbol{V}_3=[0,0,1]^T$,因此入射角餘弦分別為$\cos\phi_{l1} = t_x/t_d,\cos\phi_{l2} = t_y/t_d,\cos\phi_{l3} = t_z/t_d$,而方位角$t_\beta = \arctan(t_y/t_x) = \arctan(\cos\phi_{l2}/\cos\phi_{l1})=Ie_{l2}/Ie_{l1}$。至於天頂角,此論文則是利用多項假設,例如假設$\cos\omega=\omega$以及$\cos(\omega+\pi/4)+\cos(\omega-\pi/4)=1$等條件,來使天頂角可透過$Ie_{l3}$獲得。有了出入射角之後,三維定位的部分是透過兩個放置距離已知的LED,如圖\ref{pic:hypercube_2led},利用AOA方式距離$t_d$,值得注意的是,兩LED距離越大可提高精度,但也同時會變成環境對單點的定位,不符合本研究目的。

       

        \begin{figure}[h]
            \centering
            \includegraphics[width=7cm]{ch2pic/hypercube_2led.png}
            \caption{\cite{case:hypercube}利用兩個LED達到三維定位}
            \label{pic:hypercube_2led}
        \end{figure}
        
        \qquad
        此研究針對僅有$50\times 50mm$的範圍,垂直距離也僅有$300mm$的範圍,非常小;而實驗中二維定位可達到平均誤差為1.6mm的精度,垂直距離的誤差最大來到50mm。該研究值得注意的是,他將PD封裝成一非常小的單位,完全符合易攜帶又成本低的特色,且其利用的波段並非紅外光,為少數不利用室內現成光源的LED與PD系統,使LED與PD定位不侷限於室內照明光源的應用上。

        \item[\cite{case:3d_layers}:迭代獲得三維相對定位的案例] \hfill 
        
        \qquad
        本研究的系統架構如圖\ref{pic:env_1to1},測試範圍為$2\times 2\times 2.5m$,利用單個LED與三個PD達到三維定位,其中PD組態如圖\ref{pic:config_orient},與\cite{case:hypercube}的限制類似,一樣假設朗博次方為一、目標平面與量測平面需平行。
        
        定位方法是將PD電流兩兩相除為入射角的餘弦比值,如式\ref{eqn:hypercube},於特定垂直距離的水平面上,餘弦比值呈現於圖\ref{pic:layers_2d}。因此,兩兩比較訊號強度可得比值,透過比值由圖\ref{pic:layers_2d}可取得三條線,三條線交點即為二維定位的解$t_x,t_y$,實驗中平均誤差小於3cm。三維定位的部分,則是將量測空間垂直距離每十公分切為一層如圖\ref{pic:layers_3d},每層皆利用二維定位的方法計算一次$t_x,t_y$,再將該位置$t_x,t_y$以光傳遞模型計算出各PD理論上該得到的電流值,將計算出的電流與實際PD電流比對,最接近的該層則為三維定位的垂直距離。

         \begin{equation}
            \label{eqn:hypercube}
            \begin{aligned}
                    \frac{Ie_{11}}{Ie_{12}}= 
                \frac{
                    ( ^{P}\boldsymbol{V}_1 \cdot 
                    ^{PL}\boldsymbol{T}
                    )
                 } 
                    {
                        ( ^{P}\boldsymbol{V}_2 \cdot 
                                ^{PL}\boldsymbol{T}
                        )
                    }=\frac{\cos\phi_{11}}{\cos\phi_{12}}
            \end{aligned}
        \end{equation}

        \begin{figure}[h]
            \centering
            \includegraphics[width=12cm]{ch2pic/layers_2d.png}
            \caption{地面上PD電流強度比值(a)第一與第二個PD(b)第二與第三個PD(c)第一與第三個PD\cite{case:3d_layers}}
            \label{pic:layers_2d}
        \end{figure}

        \begin{figure}[h]
            \centering
            \includegraphics[width=12cm]{ch2pic/layers_3d.png}
            \caption{\cite{case:3d_layers}將量測空間切為多層}
            \label{pic:layers_3d}
        \end{figure}

        \qquad
        此方法特殊在其利用與絕大多數研究相同的系統架構,卻透過特殊的演算法,利用迭代各層計算與實際量測的電流大小差異,進行三維定位,且有進行實驗驗證。然而,該方法在垂直距離的定位,解析度為10cm,若要達到更高的解析度則需要切更多層,而迭代且取最小值的演算法已經十分耗時,切更多層更會加大運算負擔。

        \item[\cite{case:cart3d}:事先校正參考點以獲得三維定位的案例] \hfill 
        
        \qquad
        本研究架構與PD組態於圖\ref{pic:case_cart3d},其利用單LED與五個PD,量測範圍為$1\times 1\times 1.5m$,限制則一樣是目標平面與觀察平面需平行,雖然PD朗博次方仍假設為一,但LED的朗博次方有被考慮。二維定位的方法也是利用兩兩PD的電流比值,透過已知的PD組態回推目標物的方位,實驗達到平均$2.32cm$的經度。垂直距離則需要透過一參考點,該參考點需與目標物的垂直距離相同且已知方位,實驗達到平均$5.08cm$的精度。

        \begin{figure}[h]
            \centering
            \includegraphics[width=10cm]{ch2pic/case_cart3d.png}
            \caption{\cite{case:cart3d}研究的(a)系統架構(b)PD組態}
            \label{pic:case_cart3d}
        \end{figure}

        該研究特色之一為其考慮了LED的朗博次方,而演算法則相較簡單,透過幾何關係計算且具有實驗驗證。雖然該研究成功獲得三維定位,但卻需要有一同樣垂直距離的參考點,使得應用情境仍受限制;若移動、改變垂直距離則參考點作廢,需重新量測參考點,使用情境上受限。

    \end{description}

    以上介紹了LED與PD系統中,三篇利用單點對單點,達到三維定位的方法。各自使用的限制與演算法不同,共同的限制為PD朗博次方皆簡化為一,且目標平面與量測平面需平行。







    \subsection{LED與PD定位系統所遇困難}
    \label{chp:LEDPD_problem}
        
    綜合\ref{chp:LEDPD_restrict}章中敘述的限制與案例,以下統整,分項探討此領域所欠缺的面向。
    
    \begin{description}

        \item[- 系統限制多]\hfill 
        
        \qquad
        參考\ref{chp:LEDPD_restrict}章,常見的限制包含忽略朗博次方的影響、限制硬體選擇、限制PD組態、限制定位維度、限制使用情境等。當一次使用所有限制,即可得到式\ref{eqn:model_parallel_restrict}中簡單的模型,能夠輕易的取得相對定位。然而如\ref{chp:LEDPD_restrict}章所述,忽略朗博次方不符合實際情況,而其他的限制都會限制系統靈活度,也就是本論文最主要的目標,因此提出一限制較少的定位演算法有其必要性。
        
        \item[- 較少硬體實驗驗證]\hfill 
        
        \qquad
        在LED與PD系統中的研究,大多僅使用模型模擬,實際架設硬體進行實驗的案例較少,於\cite{survey_light2018}中整理的35個研究中,僅有不到一半的研究有進行實驗,其中\cite{case:hypercube}\cite{case:3d_layers}為有進行硬體實驗的少數研究。大多研究不進行實驗驗證的原因,其中之一為光通訊相較無線電波為較新的技術,LED光通訊在2000年被提出\cite{vlc},因此光通訊的硬體搭建並不像發展超過百年的無線電波技術完全,光通訊使用的驅動器與LED尚未被組合並商品化\cite{vlc_adv},相較無線電播技術中觸手可及的硬體,系統架設難度較高。因此大多研究先透過模擬,探索演算法與不同系統組態的可能性,具有架設完整光通訊系統能力的實驗室才會進一步進行實驗。


        \item[- 忽略誤差]\hfill 

        \qquad
        首先,最常被忽略的是多重路徑傳輸(圖\ref{pic:multipath})所產生的誤差,因為多重路徑傳輸與環境、障礙物的位置與材質皆相關,十分複雜,而該誤差於障礙物附近最為嚴重,如室內的角落。實驗上,經常將測試範圍選擇於一空曠廣大的室內中心,將測試範圍與牆壁的距離拉開,減少多重路徑傳輸效應。而僅進行模擬的研究中,模擬多重路徑誤差的難度很高\cite{multipath},三維空間中要判斷透過多次反彈能夠到達PD的光束,計算非常複雜,且針對具有障礙物等較為複雜的環境,多重誤差的模擬則更為複雜,

        再來,光定位中另一個造成誤差主要原因為環境光源,實際上在光通訊系統中,由於環境光源並不會閃爍,因此可透過將偏差(Bias)去除,以此處理環境光源。然而環境光源仍有可能使PD電流超過飽和,因此常見處理方式為將系統架設於無窗戶之室內,模擬上則常常直接將其忽略。

    \end{description}






\section{結論}
\label{chp:2_conclusion}

利用LED與PD的技術且使用幾何方法類型演算法的文獻,由於光傳遞模型變數除了距離還有出入射角,具有靈活應用的潛力,近年來開始被討論。然而,此領域有以下不足:常見的系統限制多、較少硬體實驗驗證、以及忽略誤差。本研究目標為靈活應用,因此主要著重於減少系統限制,\ref{chp:LEDPD_restrict}章中敘述的限制中,本論文目標將朗博次方完整考慮,並解決使用情境與定位維度的限制。

因此,本研究主要針對部分缺點進行改進:首先,本研究將考量LED與PD的朗博次方,完整將硬體不同的輻射模式納入模型;另外,本研究於\ref{chp:3}章中提出一能夠達到三維定位,且不需限制目標物與量測者平行的演算法,以提升使用上的靈活度,達到兩單位自由移動且能定位的目標情境。


