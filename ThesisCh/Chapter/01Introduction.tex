\chapter{緒論}

% 標題: (近紅外光)任兩點 LED與光電二極體 室內 任兩點 三維相對位置 最佳化組態

\section{前言} %(工業4.0的篇幅)

隨著工業4.0的發展,機器、人與環境之間的交互互動愈發頻繁,萬物互連的背景之下,各領域對於量測資訊的需求大量增加,其中了解位置資訊為機器與人類進行判斷與計算的基礎。若能掌握空間中某特定物與自己的相對位置資訊,則可幫助新型載具、機械手臂與人類進行決策與執行任務,例如載具了解其他載具的位置、飛行器與遙控計之間的定位、智能載具與照護目標物的互動、機械手臂與夾取目標物的定位等。綜上所述,獲取兩物之間的相對位置資訊,有其必要性。

現今室外定位主要仰賴全球衛星定位系統(GPS),然而礙於衛星訊號受建築物體遮蔽的特性,GPS定位技術無法應用在室內場域,因此發展一有效的室內定位方法獲得許多討論與研究關注。室內定位主要面對的困難與室外不同,較多的障礙物、牆壁、人員物體的密集度使多重路徑傳輸(Multipath propagation)影響大,也使訊號衰減與散射嚴重,以上議題都會增加誤差,而室內應用所需求的精度,


\subsection{室內定位簡介}
% 室內定位的簡介:(範圍對點、點對點、應用情境)

% [室內定位小小簡介總結]

在選擇合適的技術與硬體進行室內定位時,有非常多面向與設計參數需要考慮,以下條列出:

室內定位的需求與特色有許多不同面向,包含精度、覆蓋範圍、偵測距離、成本(設備、能耗)、非視線範圍內的定位(NLoS)、即時應用、設施架設範圍、對目標物與對環境的理解和掌握程度等。(可能改成條列式、配上)

礙於如此多的特性與面向,一個面面俱到的方案是不存在的。因此在設計系統時,了解不同做法的優缺點,並對自己的需求有足夠理解,進而對不同設計參數做出取捨,是完成有效室內定位系統的關鍵之一。


\section{研究動機}
\subsection{問題描述}
% 問題描述:(針對室內定位)

前言所述之室內定位發展多元,本論文針對「一移動物體對某特定移動物體的相對定位」進行探討,以下將應用情境以實際例子具體化:

\subsection{情境描述}
智能載具與服務目標的定位、輔助視障者理解移動方向、機場內針對什麼的量測 
%[看最後能不能凹到一開始發想是在醫療器具上,結合實驗室研究,這樣可見光就變更合理了]

在此情境下的量測方法有以下需求:有一易於安裝的量測單位與被量測單位,能夠快速進行拆裝,兩者皆安裝完成後即可進行相對位置量測,因此其需具備體積小、安裝方便、能耗小、成本低、不能對環境有過多的理解。

雖然室內定位這個領域已經有許多文獻探討,然而針對此情境仍沒有一個合適的方案,因此研究目的歸納如下:


\section{研究目標}


目標:低成本、不受環境影響、可分辨目標物、快速

\begin{itemize} 
    \item 發展一靈活度高,能夠套用在不同場域與情境的室內定位方法,其中場域需包含醫療環境,因此著重在探討光波段的定位應用。  
    \item 針對光波段定位,將被簡化的參數納入考量,並將組態上的限制放開,且試圖將定位維度提升到三維。
    \item 將不同應用情境納入考量,發展一套完整流程,針對不同情境進行最佳化,以提供最佳組態。
\end{itemize}

本研究以LED與PD的近紅外光定位為主,針對不同情境對LED與PD的組態與配置最佳化,其中在模擬中更完整的考慮各種因素並減少組態上的限制,以更貼近實際應用上的狀況。

\section{論文架構}
本研究分為六個章節,論文架構如圖:

[補論文架構圖]

\begin{description}
    \item[第一章] 緒論
    
    介紹研究主題,並描述本研究欲解決的問題與研究目的。
    
    \item[第二章] 文獻回顧
    
    介紹室內定位的技術與方法,並針對光定位的相關方法與現今文獻進行探討。
    
    \item[第三章] LED與PD定位方法
    
    詳細說明本研究如何利用LED與PD進行相對定位量測,並進行誤差分析。
    
    \item[第四章] 最佳化
    
    建立針對組態與硬體參數的最機化問題,並提出一流程以針對不同量測情境與工況進行最佳化。
    
    \item[第五章] 案例
    
    針對不同情境(Scenario)進行最佳化,提出最佳解並探討成效。
    
    \item[第六章] 結論
    
    整理本研究之結果討論,並敘述後續研究之方向。
    
    \end{description}







