\chapter{緒論}

% 標題: (近紅外光)任兩點 LED與光電二極體 室內 任兩點 三維相對位置 最佳化組態

\section{前言} %(工業4.0的篇幅)

隨著工業4.0的發展,機器、人與環境之間的交互互動愈發頻繁,萬物互連的背景之下,各領域對於量測資訊的需求大量增加,其中了解位置資訊為機器與人類進行判斷與計算的基礎。若能掌握空間中某特定物與自己的相對位置資訊,則可幫助新型載具、機械手臂與人類進行決策與執行任務,例如載具了解其他載具的位置、飛行器與遙控計之間的定位、智能載具與照護目標物的互動、機械手臂與夾取目標物的定位等。綜上所述,獲取兩物之間的相對位置資訊,有其必要性。

% 補iot的圖




\subsection{室內定位簡介}
% 室內定位的簡介:(範圍對點、點對點、應用情境)
現今室外定位主要仰賴全球衛星定位系統(GPS),然而礙於衛星訊號受建築物體遮蔽的特性,GPS定位技術無法應用在室內場域,因此發展一有效的室內定位方法獲得許多討論與研究關注。室內定位主要面對的困難與室外不同,較多的障礙物、牆壁、人員物體的密集度使多重路徑傳輸(Multipath propagation)影響大,也使訊號衰減與散射嚴重,以上議題都會增加誤差,且室內應用所需求的精度大多高於GPS,因此如
何設計合適的室內定位系統近年來受到研究矚目\cite{survey_indoor2014}。

室內定位最常見的分類方式為技術(Technology)與方法(Technique)\cite{survey_indoor2018},技術針對所使用的訊號與硬體種類進行分類,例如相機、紅外線、WiFi、藍芽等不同技術;而方法則是探討不同訊號皆收與處理的方式,如RSS(Received Signal Strength)與ToF(Time-of-Flight)兩種訊號接收方式,以及三邊量測法(Trilateration)、三角測量法(Triangulation)等取得相對位置訊號的方法。

\newpage

在選擇合適的技術與方法以設計室內定位系統時,需考量許多面向:

\begin{description}
    \item[量測範圍與精度] 
    包含精度、覆蓋範圍(Coverage)、目標偵測距離
    \item[成本] 
    包含硬體設備成本、系統能耗等
    \item[靈活度] 
    硬體大小、拆裝方便性、所需校正時間
    \item[是否定位可視範圍外] 
    若要進行非可視範圍(NLoS)內的定位,需利用可穿透障礙物的訊號,並犧牲精度。
    \item[是否進行即時應用] 
    若要進行即時應用,量測數據處理速度需夠快以避免訊號延遲,需犧牲訊號處理的複雜度及其附加的精度提升可能性。
    \item[目標物是否為特定物] 
    若目標物為特定某物體,則系統需有分辨訊號發送者的能力,例如無線電波的加密技術,反之光達僅有分辨訊號的有無,難以進行目標物辨識。
    \item[目標物式否有欲先安裝之特徵點] 
    是否能對目標物預先安裝特徵點也會影響系統設計,無法預先安裝特徵點的例子為自駕車,其需在陌生環境中以相機辨識周遭物體為人車或是建築物並進行定位;反例則為安裝在欲追蹤物體上的Airtag,其發送訊號以利手機追蹤並定位。 
    \item[對環境的理解程度]    
    大多WiFi與藍芽技術使用指紋比對(Fingerprinting)的方法,將系統設置完成後,預先進行大量的訊號蒐集,將接收訊號與數據庫比對得出位置,然而此方法並不能適應環境的改變,但凡環境與系統異動則原數據庫失效。  
\end{description}


礙於如此多的特性與面向,一個面面俱到的方案是不存在的。因此在設計系統時,了解不同做法的優缺點,並了解系統目標情境與需求,進而對不同設計參數做出取捨,是完成有效室內定位系統的關鍵之一\cite{survey_indoor2018}。





\section{研究動機:情境描述}
% 前言
室內定位的方法分成許多種,根據不同的應用需求與特性適用的方式也不盡相同。本研究主要目標為研究一移動物對另一特定物的相對定位方法,希望能將感測器與訊號發射器包裝成安裝方便的單位,提供一量測單位與一被量測單位,能夠靈活的將兩單位各自安裝在量測物與目標物上,在不同場域下進行三維的相對定位量測。

為更具體呈現目標的使用情境,以下舉實例描述:
\begin{description}
    \item[智慧工廠] (待補)
    \item[智慧病房] (待補)
    \item[其他]  
    智能載具與服務目標的定位、輔助視障者理解移動方向、機場內針對各種載具與行李運送的量測
\end{description}

% 補實例圖




%[看最後能不能凹到一開始發想是在醫療器具上,結合實驗室研究,這樣可見光就變更合理了]

\section{研究目的}

雖然室內定位這個領域已經有許多文獻探討,然而針對此情境仍沒有一個合適的方案,因此研究目的歸納如下:

% 目標:低成本、不受環境影響、可分辨目標物、快速

\begin{itemize} 
    \item 發展一靈活度高,能夠套用在不同場域與情境的室內定位方法,其中場域需包含醫療環境,因此著重在探討光波段的定位應用。  
    \item 針對光波段定位,將被簡化的參數納入考量,並將組態上的限制放開,且將定位維度提升到三維。
    \item 將不同應用情境納入考量,發展一套完整流程,針對不同情境進行最佳化,以提供最佳組態。
\end{itemize}

% 補想像圖

本研究以LED(發光二極體,Light Emitting Diode)與PD(光電二極體,Photodiode)的近紅外光定位為主,針對不同情境對LED與PD的組態與配置最佳化,其中在模擬中更完整的考慮各種因素並減少組態上的限制,以更貼近實際應用上的狀況。






\section{論文架構}
本研究分為六個章節,論文架構如圖:

[補論文架構圖]

\begin{description}
    \item[第一章] 緒論
    
    介紹研究主題,並描述本研究欲解決的問題與研究目的。
    
    \item[第二章] 文獻回顧
    
    介紹室內定位的技術與方法,並針對光定位的相關方法與現今文獻進行探討。
    
    \item[第三章] LED與PD定位方法
    
    詳細說明本研究如何利用LED與PD進行相對定位量測,並進行誤差分析。
    
    \item[第四章] 最佳化
    
    建立針對組態與硬體參數的最機化問題,並提出一流程以針對不同量測情境與工況進行最佳化。
    
    \item[第五章] 案例
    
    針對不同情境(Scenario)進行最佳化,提出最佳解並探討成效。
    
    \item[第六章] 結論
    
    整理本研究之結果討論,並敘述後續研究之方向。
    
    \end{description}







